\documentclass[a4paper,12pt]{article}
\usepackage[top=1in,bottom=1in,left=1.25in,right=1.25in]{geometry}
\usepackage[swedish]{babel}
\usepackage[utf8]{inputenc}
\usepackage[T1]{fontenc}
\usepackage{mathtools}
\usepackage{listings}
\usepackage{multirow}
\usepackage{graphicx}
\usepackage{hyperref}
\usepackage{amsmath}
\usepackage{amssymb}
\usepackage{framed}
\usepackage{xcolor}
\usepackage{url}
\author{Yue Jiao \\ 911024-7799}
\title{ {\Large \textbf{ Moderna fysik \\ Inlämningsuppgift 3} }}
\date{}
\begin{document}
\maketitle
Först kan vi räkna radie av böjning i magnetifältet. 
Vi kalla radie av börjning som $R$ och då kan vi skriva följande formel.
$$
(R-b)^2 + L^2 = R^2
$$
Så kan vi räkna fram $R$. 


\end{document} 
