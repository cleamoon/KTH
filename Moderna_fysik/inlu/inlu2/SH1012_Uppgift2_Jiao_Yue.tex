\documentclass[a4paper,12pt]{article}
\usepackage[top=1in,bottom=1in,left=1.25in,right=1.25in]{geometry}
\usepackage[swedish]{babel}
\usepackage[utf8]{inputenc}
\usepackage[T1]{fontenc}
\usepackage{mathtools}
\usepackage{listings}
\usepackage{multirow}
\usepackage{graphicx}
\usepackage{hyperref}
\usepackage{amsmath}
\usepackage{amssymb}
\usepackage{framed}
\usepackage{xcolor}
\usepackage{url}
\author{Yue Jiao \\ 911024-7799}
\title{ {\Large \textbf{ Moderna fysik \\ Inlämningsuppgift 2} }}
\date{}
\begin{document}
\maketitle
Enligt relativt Dopplereffekt vet vi att:
$$
\frac{f_s}{f_o} = sqrt{\frac{1-\beta}{1+\beta}}
$$
Där $f_s$ är frekvensen av ljuset som kommer från källen. 
Och $f_o$ är frekvensen av ljuset som observer se. 
$\beta$ är den relativitiska faktor som ges av bilens hastighet. 
Så vi kan anta att bilens hastighet är $v$. 
Då är $\beta = \frac{v}{c}$, där $c$ är ljusets hastighet. 

Nu kan vi kolla på tabellen av ljusets frekvenser av olika färger. 
Enligt Georgia State Universitys hemsidan får vi följande tabellen. 
\url{http://www.gsu.edu/} 
{
  \centering
  \begin{tabular}[t]{ | c | c | }
    \hline
    Färg & Frekvens (THz) \\
    \hline
    Röd & 430 \\
    Orange & 480 \\
    Gul & 510 \\
    Grön & 540 \\
    Blå & 610 \\
    Violet & 670 \\
    \hline  
  \end{tabular}
}\\

Så en röd lampa har frekvensen 430THz. 
Och då kan vi räkna fram hastigheten det krävs för att få röda ljus se grön ut, dvs frekvensen vi får bli 540THz. 
Vi vet: 
$$
\frac{f_s}{f_o} = \sqrt{\frac{1-\beta}{1+\beta}}
$$
Och vi stoppar in väder vi vet då blir det:
$$
\frac{430THz}{540THz} = \sqrt{\frac{1-\beta}{1+\beta}}
$$
Då får vi att 
$$
\beta = 22,4\%
$$
Så $\frac{v}{c} = 22,4\%$, dvs hastigheten av bilen måste vara omkring $22,4\%$ av ljusets hastighet. 

Om lampa visar grönt färg, dvs om $f_s$ blir 540THz och $\beta$ håller lika. 
Då kan vi räkna fram den nya $f_o$. 
$$
\frac{540THz}{f_o} = \sqrt{\frac{1-\beta}{1+\beta}} = \frac{430THz}{540THz}
$$
Då får vi att $f_o = 678THz$. Det är en violet ljus. 

Vi antar att det finns en bil som har massa $m$ och bränsle har massa $M$. 
Och vi kan antar att hela massenergi omvandlas till kinetiska energi. 
\end{document} 
