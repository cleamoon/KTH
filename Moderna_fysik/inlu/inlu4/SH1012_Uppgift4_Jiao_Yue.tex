\documentclass[a4paper,12pt]{article}
\usepackage[top=1in,bottom=1in,left=1.25in,right=1.25in]{geometry}
\usepackage[swedish]{babel}
\usepackage[utf8]{inputenc}
\usepackage[T1]{fontenc}
\usepackage{mathtools}
\usepackage{listings}
\usepackage{multirow}
\usepackage{graphicx}
\usepackage{hyperref}
\usepackage{amsmath}
\usepackage{amssymb}
\usepackage{framed}
\usepackage{xcolor}
\usepackage{url}
\author{Yue Jiao \\ 911024-7799}
\title{ {\Large \textbf{ Moderna fysik \\ Inlämningsuppgift 4} }}
\date{}
\begin{document}
\maketitle
\section{\textbf{a}}
För att räkna fram A och D behöver vi använda några egenskaper av vågfunktionen. 
Först är det kontinuelig och deras derivatan är också kontinuelig vid ändlig potential. 
Och integralen av vågfunktionen från -$\infty$ till $\infty$ är 1. 
Med dem kan vi räkna A och D.

Kontinulitet:
$$
 0 = A sin(k0)
$$
$$
 A sin(kL) = D exp(-\alpha x)
$$
Och derivator av vågfunktioner blir
$$
\begin{equation}
\psi (x) = 
\begin{cases}A k cos(kx)   &\mbox{0 < x < L} \\
-D \alpha exp(-\alpha x) &\mbox{x \geq L}
\end{cases}
\end{equation}



\end{document} 
