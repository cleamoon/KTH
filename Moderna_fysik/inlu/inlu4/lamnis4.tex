\documentclass[12pt,a4paper]{article}

\usepackage[utf8]{inputenc}
\usepackage[T1]{fontenc}
\usepackage[swedish]{babel}
\usepackage{amsmath}
\usepackage{amsfonts}
\usepackage{ae}
\usepackage{units}
\usepackage{icomma}
\usepackage{color}
\usepackage{graphicx}
\usepackage{bbm}
\usepackage{upgreek}
\usepackage{amsthm}
\usepackage{fullpage}
\usepackage{mathtools}
\usepackage{psfrag}
\usepackage{caption}
\setlength{\parindent}{0cm}
\newtheorem{definition}{Definition}
\newtheorem{theorem}{Theorem}
\newtheorem{proposition}{Proposition}

\begin{document}

\title{SH1012 Modern fysik \\
Inlämningsuppgift 4}
\author{Hannes Lindström 910624-1996 \\
Alias 286190}
\date{2013-03-19}
\maketitle
\thispagestyle{empty}

\clearpage
\thispagestyle{empty}
\hfill
\clearpage

\newpage
\setcounter{page}{1}

\section*{Uppgift 1}

Vi betraktar en partikel med energi $E$ bunden i en halvoändlig potentialbrunn

\begin{equation*}
U(x)=
\begin{cases}
\infty & x\le0 \\
0 & 0<x<L \\
U_0 & x\ge L
\end{cases}
\end{equation*}

där $E<U_0$. Vi är givna att partikeln uppfyller vågfunktionen

\begin{equation*}
\psi(x)=
\begin{cases}
0 & x\le0 \\
A\sin kx & 0<x<L \\
De^{-\alpha x} & x\ge L
\end{cases}
\end{equation*}

Under härledningen av denna funktion används två av fyra möjliga randvillkor: Kontinuitet i $x=0$ och släthet i $x=L$. Detta lämnar två villkor: Kontinuitet i $x=L$ samt normalisering. Dessa villkor lyder:

\begin{gather*}
\lim_{x\rightarrow L+}\psi(x)=\lim_{x\rightarrow L-}\psi(x) \\
\int_{-\infty}^{\infty}|\psi(x)|^2dx=1
\end{gather*}

Det första villkoret ger:

\begin{equation*}
A\sin kL=De^{-\alpha L}
\end{equation*}

Det andra villkoret ger:

\begin{gather*}
1=\int_0^LA^2\sin^2kxdx+\int_L^\infty D^2e^{-2\alpha x}dx=\frac{A^2}{2}\int_0^L(1-\cos2kx)dx+D^2\int_L^\infty e^{-2\alpha x}dx=\\
=\frac{A^2}{2}\left[x-\frac{\sin2kx}{2k}\right]^L_0+D^2\left[\frac{e^{-2\alpha x}}{-2\alpha}\right]_L^\infty=\frac{A^2}{2}L-\frac{A^2}{2}\frac{\sin2kL}{2k}+\frac{D^2e^{-2\alpha L}}{2\alpha}
\end{gather*}

Tillsammans följer av villkoren:

\begin{gather*}
D^2=\frac{A^2\sin^2kL}{e^{-2\alpha L}}=\frac{2\alpha}{e^{-2\alpha L}}\left(1-\frac{A^2}{2}L+\frac{A^2}{2}\frac{\sin2kL}{2k}\right) \\
A^2\left(\sin^2kL+\alpha L-\frac{\alpha\sin2kL}{2k}\right)=2\alpha
\end{gather*}

Vi tar de positiva lösningarna:

\begin{gather*}
A=\sqrt{\frac{2}{L-\frac{\sin2kL}{2k}+\frac{\sin^2kL}{\alpha}}} \\
D=\frac{\sin kL}{e^{-\alpha L}}\sqrt{\frac{2}{L-\frac{\sin2kL}{2k}+\frac{\sin^2kL}{\alpha}}}
\end{gather*}

\section*{Uppgift 2}

Vi är givna att kvantiseringsvillkoret för partikeln ovan är givet av:

\begin{equation*}
\sqrt{E}\cot\left(\frac{\sqrt{2mE}}{\hbar}L\right)=-\sqrt{U_0-E}
\end{equation*}

där

\begin{equation*}
U_0=\frac{Nh^2}{2mL^2}=4\pi^2\frac{N\hbar^2}{2mL^2}
\end{equation*}

Låt 

\begin{equation*}
k=\frac{\sqrt{2mE}}{\hbar}
\end{equation*}

Då gäller att

\begin{equation*}
U_0=4\pi^2\frac{EN}{k^2L^2}
\end{equation*}

och därmed

\begin{equation*}
f(kL):=\cot(kL)+\sqrt{4\pi^2\frac{N}{k^2L^2}-1}=0
\end{equation*}

Rötterna till $f$ svarar mot bundna tillstånd. Vi finner dem grafiskt och numeriskt för $N$=1, 2 och 4:

\begin{figure*}[h!]
\centering
\includegraphics[scale=0.75]{plotta1.eps}
\caption*{$N=1$}
\end{figure*}

\newpage

\begin{figure*}[h!]
\centering
\includegraphics[scale=0.75]{plotta2.eps}
\caption*{$N=2$}
\end{figure*}

\begin{figure*}[h!]
\centering
\includegraphics[scale=0.75]{plotta4.eps}
\caption*{$N=4$}
\end{figure*}

\begin{center}
\begin{tabular}{|l|l|l|l|l|}
\hline
N & Rot 1 & Rot 2 & Rot 3 & Rot 4 \\
\hline
1 & 2.67980 & 5.28408 & - & - \\
2 & 2.81879 & 5.60118 & 8.23814 & - \\
4 & 2.90806 & 5.80316 & 8.66403 & 11.4251 \\
\hline
\end{tabular}
\end{center}

Givet $m$ och $L$ kan dessa översättas till $E$ via sambandet

\begin{equation*}
E=\frac{k^2\hbar^2}{2m}
\end{equation*}

Med $m=m_e=9.10938188\cdot10^{-31}$ kg och $L=1$ Å blir resultatet av $E$ i eV följande:
\begin{center}
\begin{tabular}{|l|l|l|l|l|}
\hline
N & Rot 1 & Rot 2 & Rot 3 & Rot 4 \\
\hline
1 & 27.3607 & 106.380 & - & - \\
2 & 30.2724 & 119.531 & 252.571 & - \\
4 & 32.2202 & 128.307 & 285.997 & 497.327 \\
\hline
\end{tabular}
\end{center}

Om vi låter $N\rightarrow\infty$ övergår kvantiseringsvillkoret i:

\begin{equation*}
\cot{kL}=-\infty
\end{equation*}

Detta har lösningarna $kL=\pi n$ där $n\in\mathbb{Z}$. Alltså är kvantiseringsvillkoret i oändligheten:

\begin{equation*}
\frac{\sqrt{2mE}}{\hbar}=\frac{\pi n}{L}
\end{equation*}

eller

\begin{equation*}
E=\frac{\pi^2n^2\hbar^2}{2mL}
\end{equation*}

\section*{Uppgift 3}

Vi söker ett villkor för $U_0$ sådant att inget bundet tillstånd är tillåtet. Till att börja med försöker vi att finna ett villkor för $N$ sådant att detta är uppfyllt. På samma sätt som tidigare försöker vi finna rötterna till funktionen $f$ i $kL$. Om sådana rötter existerar, så existerar även bundna tillstånd. Genom att successivt pröva för mindre värden på $N$, visar det sig att $N=\frac{1}{16}$ är det största värdet på $N$ sådant att ingen rot existerar:

\newpage

\begin{figure*}[h!]
\centering
\includegraphics[scale=0.75]{plotta00625.eps}
\caption*{$N=\frac{1}{16}$}
\end{figure*}

Vi ställer oss nu frågan var denna gräns kommer ifrån. För ett bundet tillstånd gäller att $E<U_0$. Betrakta Heisenbergs osäkerhetsprincip:

\begin{equation*}
\Delta p\Delta x\ge\frac{\hbar}{2}
\end{equation*}

där $p$ är partikelns moment och $x$ är partikelns position. Om vi förutsätter att partikeln är bunden, kan vi låta $\Delta x=L$ vara osäkerheten i position. Osäkerheten i moment blir då:

\begin{equation*}
\Delta p=\frac{\hbar}{2L}
\end{equation*}

Vi förutsätter vidare att partikelns massa $m$ är känd. En undre gräns för den tillhörande osäkerheten i energin ges av uttrycket för den minimala energin, som är kinetisk:

\begin{equation*}
\Delta E=\frac{\Delta p^2}{2m}=\frac{\hbar^2}{8mL^2}
\end{equation*}

Uppenbarligen kan nu ej gälla att $\Delta E>U_0$ om partikeln ska ha bundna tillstånd, så:

\begin{equation*}
U_0\ge\Delta E=\frac{\hbar^2}{8mL^2}=\frac{h^2}{32\pi^2mL^2}
\end{equation*}

Eftersom att $\frac{Nh^2}{2\pi^2mL^2}=U_0$ ser vi mycket riktigt att för bundna tillstånd krävs att:

\begin{equation*}
N\ge\frac{1}{16}
\end{equation*}

\end{document}