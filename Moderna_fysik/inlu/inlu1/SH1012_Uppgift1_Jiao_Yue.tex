\documentclass[a4paper,12pt]{article}
\usepackage[top=1in,bottom=1in,left=1.25in,right=1.25in]{geometry}
\usepackage[swedish]{babel}
\usepackage[utf8]{inputenc}
\usepackage[T1]{fontenc}
\usepackage{mathtools}
\usepackage{listings}
\usepackage{multirow}
\usepackage{graphicx}
\usepackage{hyperref}
\usepackage{amsmath}
\usepackage{amssymb}
\usepackage{framed}
\usepackage{xcolor}
\usepackage{url}
\author{Yue Jiao \\ 911024-7799}
\title{ {\Large \textbf{ Moderna fysik \\ Inlämningsuppgift 1} }}
\date{}
\begin{document}
\maketitle
Vi vill räkna vad är andelen av myoner som kan komma hela väg ner till havsytan från Mount Everests topp.
Då behöver vi hastighet av myoner, hur snabbt myoner sönderfallar och hur hög Mount Everest är.
Hastighet av myoner gers i uppgiften. Det är 0,9$c$ där $c$ är ljus hastighet.
Enligt NE är höjden på Mount Everest $8848$ meter.
(\url{http://www.ne.se/uppslagsverk/encyklopedi/lång/mount-everest})
Och Enligt Wolfram alpha är medellivslängd av myoner $2,19704\times 10^{-6}$ sekonder. 
(\url{http://www.wolframalpha.com/input/?i=+muon})
För medellivslängd menar man att om vi räkna andelen som existera efter en vis tid $t$ blir det:
$$A = e^{-\frac{t}{\tau}}$$
Där $\tau$ är medellivslängden av myoner och $A$ blir då den andelen som finns kvar. 

\section{Om vi räkna med relativiska effekten}
Vi först kan räkna den Lorentsfaktorn $\gamma$.
$$\gamma = \frac{1}{\sqrt{1-(\frac{v}{c})^2}}$$
Där $v$ är hastigheten av myoner som är lika med $0,9c$.
Då blir Lorentsfaktorn:
$$\gamma = \frac{1}{1-0,9^2} = 2,29416$$
Om vi ställa oss i myonernas referensram är hela sträckan kortare på grund av relativiska effekten.
Den originella sträckan är höjden av Mount Everest som lika med $h = 8848$ meter.
Det blir
$$h_m = \frac{h}{\gamma} = \frac{8848}{2.29416} = 3856,75$$
meter.
Och hastigheten av myoner är 
$$v = 0,9\times c = 0,9 * 3\times 10^8 = 2,7 \times 10^8$$ 
meter per sekund för att ljusets hastighet är $3\times 10^8$ enligt NE  
(\url{http://www.ne.se/uppslagsverk/encyklopedi/enkel/ljushastigheten}) .

Då blir total restiden 
$$t = \frac{h_m}{v} = \frac{3856,75}{2,7\times 10^8} = 1,42843\times 10^{-5}$$ 
sekunder. 
Då blir 
$$A = e^{-\frac{t}{\tau}} = e^{-\frac{1,42843\times 10^{-5}}{2,19704\times 10^{-6}}} = 0,150105\% = 0.15\%$$
Så ungefär $0.15\%$ av alla myoner överleva tills havsytan. 

\section{Om man inte tar hänsyn till relativiska effekten}
Det blir en enkel räkning.
Hela sträckan är $h = 8848$ meter och hastigheten av myoner är $v = 0,9c = 2,7\times 10^8$ meter per sekunder. 
Då blir totala restiden 
$$t = \frac{h}{v} = \frac{8848}{2,7\times10^8} = 3,27703\times10^{-5} $$
sekunder. 
Då andelen som existera blir:
$$A = e^{-\frac{t}{\tau}} = e^{-\frac{3,27703\times10^{-5}}{2,19704\times10^{-6}}} = 3,32822\times10^5\% = 3,3\times10^5\%$$ 
Så ungefär $3,3\times10^5\%$ av alla myoner överleva tills havsytan om man inte tar hänsyn till relativiska effekten.

\end{document} 
