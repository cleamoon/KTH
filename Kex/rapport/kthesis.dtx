%
% \begin{abstract}
%    This document describes \LaTeX\ document classes for KTH
%    theses. More information about the graphical profile at
%    KTH is available on the web:
%    \url{http://www.kth.se/internt/projekt/grafiskprofil/}.
% \end{abstract}
%
%
%
% \section{Introduction}
%
%    The KTH thesis document classes can be downloaded from
%    \url{ftp://ftp.nada.kth.se/pub/tex/local/kthesis.tar.gz}.
%    They are built on the highly customizable document class
%    \class{memoir} written by Peter Wilson. Hence they require
%    the \class{memoir} class to work; this class comes with most
%    modern \LaTeX\ distributions and it can be
%    downloaded from any CTAN mirror. It has been verified
%    that the thesis classes work together with version
%    1.3a (2002-11-22) of the \class{memoir} class with
%    version 1.9 (2003-11-19) of the \class{memoir} patch file
%    as well as together with  version
%    1.6 (2004-01-31) of the \class{memoir} class with
%    version 2.1a (2004-02-19) of the \class{memoir} patch file.
%    The \class{kthesis} classes also require the KTH logo.
%    This logo is available in a package
%    suitable for use with \TeX\ and \LaTeX\ at
%    \url{ftp://ftp.nada.kth.se/pub/tex/local/}.
%
%    Most modern \TeX\ distibrutions have a number of directories
%    (aka folders) designated for local additions of document
%    classes and other related files. The archives containing the
%    document classes and the KTH logo should be unpacked in one
%    of theses directories with the directory structure intact.
%    With MiK\TeX, the directory for local additions is normally
%    \class{C:\textbackslash localtexmf} -- with te\TeX\ it is probably
%    easiest to unpack the archives in the folder \url{~/texmf}. Remember
%    to update the file name database when the files have been
%    installed -- in MiK\TeX\ this is done through the configuration
%    program while te\TeX\ users have to run \command{texhash} from
%    the shell prompt.
%
%
%
% \section{Producing Doctoral and Licentiate Theses}
%
%    This macro package contains two classes for
%    documents related to Doctoral and Licentiate theses:
%    \class{kthesis} and \class{kthspik}. The former is used to
%    typeset the thesis itself. Specifically, it should be used
%    to produce the \emph{insert}, i.e., all pages in the book but
%    not the cover. The latter class can be used to typeset the
%    announcement sheet that states when the defense takes place.
%    Such a sheet is called \emph{spikblad} in Swedish and it is
%    typically distributed to various departments and academic
%    institutions.
%
%    As mentioned above, the document class is an add-on to the
%    document class \class{memoir} and it hence accepts the same
%    class options and behaves in the same way as that class with
%    the following exceptions: 1)~The only paper size
%    specifications recognized are \option{a4paper} and
%    \option{g5paper}. 2)~The class redefines the
%    \verb|\maketitle| command to produce a title page containing
%    the KTH logo. 3)~It is possible to omit the standard text
%    ``Akademisk avhandling\ldots'' with the class option
%    \option{noblurb}. 4)~The additional class option
%    \option{electronic} adapts the layout of the page on the
%    stock to electronic publishing. 5)~It is possible to specify
%    a subtitle, bibliographic information and data concerning
%    the thesis defense.
%
%    A standard \LaTeX\ document class, such as the
%    \class{article} class, supports the definition of the
%    document title, author and date using the commands
%    \verb|\title|, \verb|\author| \verb|\date|. The
%    \class{kthesis} class supports those commands and many more:
%    \begin{itemize}
%    \item \verb|\title| -- the document title (mandatory). It
%      appears at the top of page~1.
%    \item \verb|\subtitle| -- the document subtitle (optional).
%      It appears below the title on page~1.
%    \item \verb|\author| -- the author of the thesis
%      (mandatory). It appears below the subtitle on page~1.
%    \item \verb|\date| -- the date of publication (mandatory).
%      It appears in the copyright notice and may be written in
%      either Swedish or English (the former is probably best
%      since most text on page~2 is already in Swedish).
%    \item \verb|\thesistype| -- the kind of thesis you are
%      writing (mandatory). It should conform on the text
%      specified at the bottom right corner of the cover, i.e.,
%      it should be something like ``Doctoral Thesis'',
%      ``Licentiate Thesis'', or ``Master of Science Thesis'',
%      and it appears near the bottom of page~1.
%    \item \verb|\imprint| -- where and when the thesis is
%      published (mandatory). It should conform on the text
%      specified at the bottom right corner of the cover, i.e.,
%      it should be something like ``Stockholm, Sweden, 2004'',
%      and it appears at the bottom of page 1.
%    \item \verb|\examen| -- the kind of degree the thesis is
%      associated with (mandatory unless you use the
%      \option{noblurb} class option). This text must be in
%      Swedish and it appears in the standard text beginning with
%      ``Akademisk avhandling'' on page~2.
%    \item \verb|\disputationsdatum| -- the date of the defense
%      (mandatory unless you use the \option{noblurb} class
%      option). This text must be in Swedish and it appears in
%      the standard text beginning with ``Akademisk avhandling''
%      on page~2.
%    \item \verb|\disputationslokal| -- the place of the defense
%      (mandatory unless you use the \option{noblurb} class
%      option). This text must be in Swedish and it appears in
%      the standard text beginning with ``Akademisk avhandling''
%      on page~2.
%    \item \verb|\publisher| -- the printing house (mandatory).
%      If you use the KTH press, this text should be
%      ``Universitetsservice US AB''.
%    \item \verb|\address| -- the address to the department producing
%      the thesis. If no address is specified, the KTH address is used.
%      May be written in either Swedish or English (the former is
%      probably best since most text on page~2 is already in Swedish).
%    \item \verb|\isbn| -- the ISBN number of the thesis (optional).
%    \item \verb|\isrn| -- the ISRN number of the thesis (optional).
%    \item \verb|\issn| -- the ISSN number of the thesis (optional).
%    \item \verb|\trita| -- the TRITA number of the thesis (optional).
%    \item \verb|\kthlogo| -- the file used for the KTH logo (optional).
%      If you want to use the KTH logo corresponding to your department,
%      specify the file name with this command. Note that the thesis is
%      only designed for the official KTH logo as defined by the KTH
%      Presidential Decisions 2002-11-11 (nr 938, dnr 930-1409-02, dossier
%      19) and 2003-03-03 (nr 181/03, dnr 930-2003-0437, dossier 19).
%    \end{itemize}
%
%    Since \class{kthesis} is built on the highly configurable
%    \class{memoir} class, it is possible to redefine the
%    document appearance with a reasonable effort. Details
%    regarding this is described in the excellent \class{memoir}
%    class manual written by Peter Wilson (\emph{The memoir class
%    for configurable typesetting}, fifth edition, August 2002).
%    This manual also discusses typographic conventions and book
%    production in general.
%
%
%
% \section{Producing Master's Theses}
%
%    This macro package contains one document class file for
%    Master's theses: \class{kth-mag}. This
%    class is used to typset the thesis itself, including a generic
%    title page. The official title page is usually produced by the
%    department responsible for the Master's project---consult your
%    department for details. Nevertheless, the title page produced
%    by the class file can be useful if you intend to put an electronic
%    version of the thesis on the web.
%
%    As mentioned above, the document class is an add-on to the
%    document class \class{memoir} and it hence accepts the same
%    class options and behaves in the same way as that class with
%    the following exceptions: 1)~The only paper size
%    specifications recognized are \option{a4paper} and
%    \option{g5paper}. 2)~The class redefines the
%    \verb|\maketitle| command to produce a title page containing
%    the KTH logo. 3)~The additional class option
%    \option{electronic} adapts the layout of the page on the
%    stock to electronic publishing. 4)~It is possible to specify
%    a subtitle and certain information that should appear on the
%    the title page.
%
%    A standard \LaTeX\ document class, such as the
%    \class{article} class, supports the definition of the
%    document title, author and date using the commands
%    \verb|\title|, \verb|\author| \verb|\date|. The
%    \class{kth-mag} class supports those commands and some more:
%    \begin{itemize}
%    \item \verb|\title| -- the document title (mandatory). It
%      appears at the top of page~1.
%    \item \verb|\subtitle| -- the document subtitle (optional).
%      It appears below the title on page~1.
%    \item \verb|\author| -- the author of the thesis
%      (mandatory). It appears below the subtitle on page~1.
%    \item \verb|\date| -- the date of publication (mandatory).
%      It appears in the copyright notice and may be written in
%      either Swedish or English (the former is probably best
%      since most text on page~2 is already in Swedish).
%    \item \verb|\blurb| -- the text that should appear near
%      the bottom of the title page. This text could typically
%      include the name of the thesis supervisor.
%    \item \verb|\trita| -- the TRITA number of the thesis (optional).
%    \item \verb|\kthlogo| -- the file used for the KTH logo (optional).
%      If you want to use the KTH logo corresponding to your department,
%      specify the file name with this command. Note that the thesis is
%      only designed for the official KTH logo as defined by the KTH
%      Presidential Decisions 2002-11-11 (nr 938, dnr 930-1409-02, dossier
%      19) and 2003-03-03 (nr 181/03, dnr 930-2003-0437, dossier 19).
%    \end{itemize}
%
%    Since \class{kth-mag} is built on the highly configurable
%    \class{memoir} class, it is possible to redefine the
%    document appearance with a reasonable effort. Details
%    regarding this is described in the excellent \class{memoir}
%    class manual written by Peter Wilson (\emph{The memoir class
%    for configurable typesetting}, fifth edition, August 2002).
%    This manual also discusses typographic conventions and book
%    production in general.
%
%
%
% \section{References and \BibTeX\ support}
%
%    This package also includes two style files for the \BibTeX\
%    program. The first one---\bst{kthplain}---is a small extension
%    of the \BibTeX\ standard style file \bst{plain}. In addition to
%    the fields supported by the \bst{plain} style file, \bst{kthplain}
%    also supports the fields  ISBN, ISSN and URL. If the \style{url}
%    package has been loaded, it is used to typset the URLs. Also, the
%    \bst{kthplain} style does not use the now obsolete font changing
%    commands that are not supported by the \class{memoir} class.
%
%    Many authors prefer to use so called ``author-year'' citation
%    systems, i.e., references to entries in the bibliography are
%    of the form (Smith, 2003). The by far best way to achieve this
%    is to use the \style{natbib} package.
%    The second bibliography style file---\bst{kthnat}---is a
%    slightly modified version of the \bst{plainnat} style that comes
%    with the \style{natbib} package. In particular, the \bst{kthnat}
%    style does not use the now obsolete font changing commands that
%    are not supported by the \class{memoir} class. The \bst{kthnat}
%    style is used in exactly the same way as \bst{plainnat}, see the
%    \style{natbib} manual for details.
%
%    The \bst{kthnat} bibliography style formats the actual bibliography
%    as a numbered list, sorted by author. More elaborate bibliography
%    styles can easily be constructed with the \style{custom-bib}
%    package that comes with most modern \LaTeX\ distributions. It is
%    an interactive program written in \LaTeX. As a user, one replies to
%    a series of questions, thereby specifying how the bibliography
%    should be formatted,
%    the corresponding bibliography style file is then constructed
%    automatically. The interactive program is started by executing
%    \command{latex makebst} in a shell command window.
%
%
%
% \section{Printing your Thesis}
%
%    KTH Doctoral and Licentiate theses are normally produced in
%    G5 format (169\,mm by 239\,mm). The \class{kthesis} class assumes
%    that the document is printed on A4 paper and then trimmed to~G5.
%    In other words, the stock size is A4 and the page size is~G5.
%    The page is centered vertically and flushed towards the spine
%    on the stock. Documents submitted for printing at the KTH press
%    should be in PostScript format and they should not contain any
%    trimming marks. However, it is advisable to also give the printing
%    house a couple of printed pages---in addition to the entire thesis,
%    submitted electronically without trimming marks---and indicate the
%    final page dimensions on those printed sheets. The
%    \class{kthesis} class can automatically show such trimming
%    marks if it is given the class option \option{showtrims}.
%    Once you receive a demo printout of your thesis from the
%    printing house, check it carefully for errors in
%    the positioning and size of the page.
%
%    As specified by the KTH graphical profile, the cover is
%    designed and produce separately. The
%    first page produced by the \class{kthesis} class
%    is not identical to the cover; it is in fact the norm
%    in book production that the cover and page~1 of the
%    insert differ in printed books. Some departments require
%    that also the first page of the insert is produced separately;
%    check your local department regulations for details.
%
%    Remember that the title of your thesis will be entered in
%    several library catalogues and, hopefully, cited in many
%    articles and theses. You should therefore use only normal
%    letters and digits in the title; avoid mathematical
%    notation.
%
%    The ``spikblad'' is normally printed slightly smaller than
%    G5 (160\,mm by 230\,mm) so that it can fit physically inside
%    the bound thesis. The \class{kthspik} class assumes that the
%    document is printed on A4 paper and then trimmed; the page
%    is centered both vertically and horizontally on
%    the stock. If you do not print and trim the announcement
%    sheets yourself, tell the press the size of the trimmed
%    pages and how the pages are laid out on the stock. Note that
%    the typearea is the same, both for the \class{kthesis} class
%    and the \class{kthspik} class. This ensures that the line
%    breaks in, \textit{i.e.}, the abstract are independent of
%    the document class.
%
%    Master's theses are usually ``printed'' locally on the
%    department copier---consult your department for details.
%
%
%
% \section{Electronic Publishing}
%
%    As mentioned above, the class option \option{electronic}
%    adapts the output to electronic publishing. This only has
%    one effect: The page is cropped to its actual size.
%    You should therefore
%    not use the \option{showtrims} class option in conjunction
%    with \option{electronic}.
%
%
%
% \section{Modifying the Layout}
%
%    Since the thesis classes are based on the highly configurable
%    \class{memoir} class, it is reasonable simple to modify the
%    layout as desired. A small example of such modifications is
%    included below.
%    \begin{macrocode}
%<*layout>
\ProvidesPackage{nada-ex}[2005/03/16]
%% Example customizations of the KTH thesis classes.
%% See the package documentation for more details.
%    \end{macrocode}
%    By default, the memoir class does not number subsections, nor are
%    they included in the table of contents. The following four lines
%    change this.
%    \begin{macrocode}
\maxsecnumdepth{subsection}
\setsecnumdepth{subsection}
\maxtocdepth{subsection}
\settocdepth{subsection}
%    \end{macrocode}
%    Sans serif fonts in chapter and section headings can look nice.
%    \begin{macrocode}
\renewcommand{\partnamefont} {\usefont{T1}{lmss}{sbc}{n}\boldmath\huge}
\renewcommand{\partnumfont}  {\usefont{T1}{lmss}{sbc}{n}\boldmath\huge}
\renewcommand{\parttitlefont}{\usefont{T1}{lmss}{sbc}{n}\boldmath\Huge}
\renewcommand{\chapnamefont} {\usefont{T1}{lmss}{sbc}{n}\boldmath\huge}
\renewcommand{\chapnumfont}  {\usefont{T1}{lmss}{sbc}{n}\boldmath\huge}
\renewcommand{\chaptitlefont}{\usefont{T1}{lmss}{sbc}{n}\boldmath\Huge}
\setsecheadstyle   {\usefont{T1}{lmss}{bx}{n}\boldmath\Large\raggedright}
\setsubsecheadstyle{\usefont{T1}{lmss}{bx}{n}\boldmath\large\raggedright}
\setparaheadstyle   {\normalsize\bfseries\boldmath}
\setsubparaheadstyle{\normalsize\bfseries\boldmath}
%    \end{macrocode}
%    Instead of slanted type in the page headers, use upright, slightly
%    smaller, type.
%    \begin{macrocode}
\makeevenhead{headings}%
  {\normalfont\small\thepage}{}{\normalfont\small\leftmark}
\makeoddhead{headings}%
  {\normalfont\small\rightmark}{}{\normalfont\small\thepage}
%    \end{macrocode}
%    We also modify the style of figure and table captions slightly.
%    \begin{macrocode}
\setlength{\@tempdima}{\textwidth}
\addtolength{\@tempdima}{-2\leftmargini}
\captionwidth{\@tempdima}
\changecaptionwidth
\captiondelim{. }
\captionnamefont{\normalfont\footnotesize\bfseries}
\captiontitlefont{\normalfont\footnotesize}
%    \end{macrocode}
%    For some reason, standard \LaTeX\ has glue with non-zero
%    stretchability in |\parskip|. Not nice\ldots
%    \begin{macrocode}
\setlength{\parskip}{0pt}
%    \end{macrocode}
%    Finally, we use the KTH logo corresponding to the School
%    of Computer Science and Communication instead of the
%    generic one.
%    \begin{macrocode}
\kthlogo{kth_svv_comp_science_comm}
%    \end{macrocode}
%</layout>
%
%
% \section{The Source Code}
%
%    This section contains the source code for the
%    \class{kthesis} document class. The bibliography
%    styles are omitted from this document but they
%    are present in the file \file{kthesis.dtx}.
%    \begin{macrocode}
%<*classfile>
\NeedsTeXFormat{LaTeX2e}[2001/01/01]
%<spikblad>\ProvidesClass{kthspik}
%<avhandling&drlic>\ProvidesClass{kthesis}
%<avhandling&exjobb>\ProvidesClass{kth-mag}
              [2004/12/14 v2.1 Document Class for KTH Theses]
%    \end{macrocode}
%    Now follows the content of the class files. First we define
%    two new if statements that are needed to handle the new class
%    options \option{noblurb} and \option{electronic}.
%    \begin{macrocode}
%<drlic>\newif\if@kthesis@blurb \@kthesis@blurbtrue
%<drlic>\DeclareOption{noblurb}{\@kthesis@blurbfalse}
\newif\if@kthesis@electronic \@kthesis@electronicfalse
\DeclareOption{electronic}{\@kthesis@electronictrue}
%    \end{macrocode}
%    Then we define two macros that setup the stock size and the
%    papersize. Depending on the page size selected with the class
%    options (either \option{a4paper} or \option{g5paper}, the latter
%    being the default), one of the two macros are called.
%    \begin{macrocode}
\def\@kthesis@setpapersize@afour{%
  \setstocksize{297mm}{210mm}
  \settrimmedsize{297mm}{210mm}{*}
  \settypeblocksize{49pc}{33pc}{*}}
\def\@kthesis@setpapersize@gfive{%
  \if@kthesis@electronic
%    \end{macrocode}
%    If the class option \option{electronic} was specified, we set
%    both the stock size and the paper size to the actual dimensions
%    of the page ($169\mbox{\,mm} \times 239\mbox{\,mm}$ for the
%    thesis and ($160\mbox{\,mm} \times 230\mbox{\,mm}$ for the
%    announcement sheet. Unless we run pdf\LaTeX\ we also insert
%    a special that specifies the page size for dvips.
%    \begin{macrocode}
%<*avhandling>
    \setstocksize{239mm}{169mm}
    \settrimmedsize{239mm}{169mm}{*}
    \ifpdf
    \else
      \special{papersize=169mm,239mm}
    \fi
%</avhandling>
%<*spikblad>
    \setstocksize{230mm}{160mm}
    \settrimmedsize{230mm}{160mm}{*}
    \ifpdf
    \else
      \special{papersize=160mm,230mm}
    \fi
%</spikblad>
%    \end{macrocode}
%    If the class option \option{electronic} was not specified, the
%    page layout should be such that the thesis can be printed and
%    bound at the KTH press. For the insert, this means that the G5
%    page should be centered vertically on the stock and flushed
%    towards the spine horizontally. Since the announcement sheet
%    is not bound, we center it on the stock. In this way, all four
%    edges of the sheet can be trimmed.
%
%    If we run pdf\LaTeX\ we need to set the page size in the PDF file
%    to the size of the stock. Since the graphics package resets the
%    PDF page size when it is loaded, we need to reset the PDF page
%    size in the ``begin document hook''.
%    \begin{macrocode}
  \else
    \setstocksize{297mm}{210mm}
%<avhandling>    \settrimmedsize{239mm}{169mm}{*}
%<spikblad>    \settrimmedsize{230mm}{160mm}{*}
    \setlength{\trimtop}{\stockheight}
    \addtolength{\trimtop}{-\paperheight}
    \setlength{\trimtop}{0.5\trimtop}
    \setlength{\trimedge}{\stockwidth}
    \addtolength{\trimedge}{-\paperwidth}
%<spikblad>    \setlength{\trimedge}{0.5\trimedge}
    \ifpdf
      \AtBeginDocument{%
        \setlength{\pdfpagewidth}{\stockwidth}
        \setlength{\pdfpageheight}{\stockheight}}
    \fi
  \fi
%    \end{macrocode}
%    No matter what, the size of the type block is always the same.
%    \begin{macrocode}
  \settypeblocksize{43pc}{30pc}{*}}
%    \end{macrocode}
%    Next comes code to handle the remaining class options. Since we
%    build the class on the existing class \class{memoir}, we just
%    pass most options to that class. Exceptions to this are the
%    options that select stock size---we don't support them and
%    instead use the options \option{a4paper} and \option{g5paper}
%    to define the page size.
%    \begin{macrocode}
\DeclareOption{g5paper}{\def\kthesis@trimmedsize{gfive}}
\DeclareOption{a3paper}{\OptionNotUsed}
\DeclareOption{a4paper}{\def\kthesis@trimmedsize{afour}}
\DeclareOption{a6paper}{\OptionNotUsed}
\DeclareOption{a5paper}{\OptionNotUsed}
\DeclareOption{b3paper}{\OptionNotUsed}
\DeclareOption{b4paper}{\OptionNotUsed}
\DeclareOption{b5paper}{\OptionNotUsed}
\DeclareOption{b6paper}{\OptionNotUsed}
\DeclareOption{letterpaper}{\OptionNotUsed}
\DeclareOption{legalpaper}{\OptionNotUsed}
\DeclareOption{executivepaper}{\OptionNotUsed}
\DeclareOption{ebook}{\OptionNotUsed}
\DeclareOption{landscape}{\OptionNotUsed}
\DeclareOption*{\PassOptionsToClass{\CurrentOption}{memoir}}
\ExecuteOptions{g5paper}
\ProcessOptions\relax
%    \end{macrocode}
%    We can now load the \class{memoir} class. Since we assume
%    that the stock is A4~paper, we specify that.
%    \begin{macrocode}
\LoadClass[a4paper]{memoir}
%    \end{macrocode}
%    We now set the page size. The macros that set page size
%    may add code to the ``begin document hook'' that resets the
%    the lengths \verb|\pdfpagewidth| and \verb|\pdfpageheight|.
%    Since it appears that the graphics package changes those
%    lengths, we need to insert a hook that loads the graphics
%    package before calling the page size macros.
%    \begin{macrocode}
\AtBeginDocument{\RequirePackage{graphics}}
\@nameuse{@kthesis@setpapersize@\kthesis@trimmedsize}
%    \end{macrocode}
%    We then finalize the layout.
%    \begin{macrocode}
\setlrmargins{*}{*}{*}
\setulmargins{*}{*}{*}
\checkandfixthelayout
%    \end{macrocode}
%    The thesis classes support some additional document information;
%    the following macros handle that.
%    \begin{macrocode}
\def\subtitle{\gdef\thesubtitle}
%<*drlic>
\def\thesistype{\gdef\thethesistype}
\def\examen{\gdef\theexamen}
\def\disputationsdatum{\gdef\thedisputationsdatum}
\def\disputationslokal{\gdef\thedisputationslokal}
\def\imprint{\gdef\theimprint}
\def\isbn{\gdef\theisbn}
\def\isrn{\gdef\theisrn}
\def\issn{\gdef\theissn}
\def\publisher{\gdef\thepublisher}
\def\address{\gdef\theaddress}
%</drlic>
\def\trita{\gdef\thetrita}
\def\kthlogo{\gdef\thekthlogo}
%<exjobb>\def\blurb{\gdef\theblurb}
%    \end{macrocode}
%    Default values\ldots
%    \begin{macrocode}
\let\thesubtitle\relax
%<*drlic>
\def\thethesistype{\ClassError{kthesis}%
  {\protect\thesistype\space not given}%
  {Please Read The Fine Manual.}}
\def\theexamen{\ClassError{kthesis}%
  {\protect\examen\space not given}%
  {Please Read The Fine Manual.}}
\def\thedisputationsdatum{\ClassError{kthesis}
  {\protect\thedisputationsdatum\space not given}%
  {Please Read The Fine Manual.}}
\def\thedisputationslokal{\ClassError{kthesis}
  {\protect\thedisputationslokal\space not given}%
  {Please Read The Fine Manual.}}
\def\theimprint{\ClassError{kthesis}%
  {\protect\theimprint\space not given}%
  {Please Read The Fine Manual.}}
\def\thepublisher{\ClassError{kthesis}%
  {\protect\thepublisher\space not given}%
  {Please Read The Fine Manual.}}
\def\theaddress{KTH\\SE-100 44 Stockholm\\SWEDEN}
%</drlic>
\def\thekthlogo{kth_svv}
%<exjobb>\let\theblurb\relax
%    \end{macrocode}
%    Define a page style that puts bibliographic information at
%    the bottom of every page. The style is used by default in
%    the \class{kthsipk} class but it is available also in the
%    \class{kthesis} class.
%    \begin{macrocode}
%<*drlic>
\makepagestyle{kthabstract}
\makeevenhead{kthabstract}{}{}{}
\makeoddhead{kthabstract}{}{}{}
\makeevenfoot{kthabstract}{}{\normalfont\scriptsize
  \def\@maybebullet{\if@bullet\space\textbullet\space\fi}
  \newif\if@bullet \@bulletfalse
  \@ifundefined{thetrita}{}{\@maybebullet\thetrita\@bullettrue}%
  \@ifundefined{theissn}{}{\@maybebullet\theissn\@bullettrue}%
  \@ifundefined{theisrn}{}{\@maybebullet\theisrn\@bullettrue}%
  \@ifundefined{theisbn}{}{\@maybebullet\theisbn\@bullettrue}%
}{}
\makeoddfoot{kthabstract}{}{\normalfont\scriptsize
  \def\@maybebullet{\if@bullet\space\textbullet\space\fi}
  \newif\if@bullet \@bulletfalse
  \@ifundefined{thetrita}{}{\@maybebullet\thetrita\@bullettrue}%
  \@ifundefined{theissn}{}{\@maybebullet\theissn\@bullettrue}%
  \@ifundefined{theisrn}{}{\@maybebullet\theisrn\@bullettrue}%
  \@ifundefined{theisbn}{}{\@maybebullet\theisbn\@bullettrue}%
}{}
%<spikblad>\pagestyle{kthabstract}
%</drlic>
%    \end{macrocode}
%    Finally, we redefine the \verb|\maketitle| command. For the
%    \class{kthesis} class, we get a different layout on the title
%    page and bibliographic data on the back of the title page.
%    For the \class{kthspik} class we get a different layout on the
%    title page and bibliographic information in the page footer.
%    The code that typesets the KTH logo assumes that the sides
%    in the black square are 369 postscript points long. This seems
%    to be the only way to reliably get a logo where black square is
%    25\,mm by 25\,mm on the title page---since the text below the square
%    is sometimes wider than the square, we cannot just set the width of
%    the entire logo to 25\,mm.
%    \begin{macrocode}
%<*drlic>
\newif\if@kthesis@babel \@kthesis@babelfalse
\AtBeginDocument{%
  \@ifpackagewith{babel}{swedish}%
     {\@kthesis@babeltrue}{\@kthesis@babelfalse}}
\def\maketitle{%
  \begingroup
  \sffamily
%<*avhandling>
  \pagestyle{empty}
  \renewcommand*{\thepage}{title-\arabic{page}}
  \noindent
  \parbox[b][0.6\textheight]{\textwidth}{\centering\noindent
%</avhandling>
%<*spikblad>
  {\centering\noindent
%</spikblad>
    \scalebox{0.192}{\includegraphics{\thekthlogo}}\par
    \vskip1cm
    \LARGE\bfseries\thetitle\par
    \vfill
    \normalsize\mdseries\thesubtitle\par
    \vfill
    \large\MakeUppercase{\theauthor}\par
%<*avhandling>
  }\par
  \vfill\noindent
  \parbox[b]{\textwidth}{\centering\noindent\large
%</avhandling>
%<spikblad>    \vfill
    \thethesistype\par
    \theimprint\par
%<*avhandling>
  }\par
  \newpage
  \hbox{}\vfill
  \rmfamily
  \noindent\begin{tabular}[b]{@{}l@{}}
    \@ifundefined{thetrita}{}{\thetrita\\}%
    \@ifundefined{theissn}{}{\theissn\\}%
    \@ifundefined{theisrn}{}{\theisrn\\}%
    \@ifundefined{theisbn}{}{\theisbn\\}%
  \end{tabular}
  \hfill
  \noindent\begin{tabular}[b]{@{}r@{}}
    \theaddress
  \end{tabular}
  \par\bigskip
%</avhandling>
%<*spikblad>
  }
  \vfill
  \rmfamily
%</spikblad>
  \if@kthesis@blurb
    \if@kthesis@babel
      \begin{otherlanguage}{swedish}
        \noindent Akademisk avhandling som med tillst\aa nd
        av Kungl Tekniska h\"{o}gskolan framl\"{a}gges till
        offentlig granskning f\"{o}r avl\"aggande av
        \theexamen\space \thedisputationsdatum\space i
        \thedisputationslokal.\par
      \end{otherlanguage}
    \else
      \noindent Aka\-de\-misk av\-hand\-ling som med
      till\-st\aa nd av Kungl Tek\-niska h\"{o}g\-skolan
      fram\-l\"{a}g\-ges till of\-fent\-lig gransk\-ning
      f\"{o}r av\-l\"ag\-gande av \theexamen\space
      \thedisputationsdatum\space i \thedisputationslokal.\par
    \fi
    \bigskip
  \fi
%<*avhandling>
  \noindent
  \copyright \space \theauthor, \thedate
  \par\bigskip\noindent
  Tryck: \thepublisher\par
  \cleardoublepage
%</avhandling>
%<spikblad>  \newpage
  \endgroup
}
%</drlic>
%<*exjobb>
\def\maketitle{%
  \begingroup
  \sffamily
  \pagestyle{empty}
  \renewcommand*{\thepage}{title-\arabic{page}}
  \noindent
  \parbox[b][0.6\textheight]{\textwidth}{\centering\noindent
    \scalebox{0.192}{\includegraphics{\thekthlogo}}\par
    \vskip1cm
    \LARGE\bfseries\thetitle\par
    \vfill
    \normalsize\mdseries\thesubtitle\par
    \vfill
    \large\MakeUppercase{\theauthor}\par
  }\par
  \vfill\noindent
  \parbox[b]{\textwidth}{\centering\noindent
    \theblurb\par
  }
  \vfill
  \centerline{\thetrita}
  \rmfamily
  \cleardoublepage
  \endgroup
}
%</exjobb>
%</classfile>
%    \end{macrocode}
%
%
%
% \section{Template Files}
%    
%    Included with the KTH thesis class are also several skeleton
%    files. The first two, \path{kth-demo.tex} and \path{kth-spik.tex},
%    use the classes \class{kthesis} and \class{kthspik}, respectively.
%    The last two, \path{kth-ex1.tex} and \path{kth-ex2.tex} use the
%    class \class{kth-mag}.
%    Note that the imprint and bibliographic information used in
%    those files may not be correct for your thesis.
%    \begin{macrocode}
%<demo>\documentclass[showtrims]{kthesis}
%<spik>\documentclass[showtrims]{kthspik}
%<ex1>\documentclass[a4paper,11pt]{kth-mag}
%<ex2>\documentclass[showtrims]{kth-mag}
%    \end{macrocode}
%    The thesises classes work best if the babel package is loaded with
%    both Swedish and English languages. Even if the thesis is written
%    in English, there are some pices of text that must be in Swedish.
%    \begin{macrocode}
%<*demo|spik|ex1|ex2>
\usepackage[T1]{fontenc}
\usepackage{textcomp}
\usepackage{lmodern}
\usepackage[latin1]{inputenc}
\usepackage[swedish,english]{babel}
%</demo|spik|ex1|ex2>
%    \end{macrocode}
%    It is reasonably simple to modify the layout of the thesis
%    since the document class is based on the memoir class.
%    \begin{macrocode}
%<ex1>\usepackage{nada-ex}
%    \end{macrocode}
%    It is sometimes convenient to keep all bibliographic information in
%    one file, this minimizes the risc that some item is corrupted.
%    \begin{macrocode}
%<*demo|spik>
\input{kth-bibl}
%</demo|spik>
%<*biblio>
\title{Lorem ipsum dolor sit amet, sed diam nonummy nibh eui
       mod tincidunt ut laoreet dol}
\subtitle{Duis autem vel eum iruire dolor in hendrerit in
          vulputate velit esse molestie consequat, vel illum
          dolore eu feugiat null}
\author{Namn Namnet}
\date{maj 2003}
\thesistype{Doctoral Thesis}
\imprint{Stockholm, Sweden 2003}
\examen{teknologie doktorsexamen i datalogi}
\disputationsdatum{torsdagen den 17 maj 2003 klockan 10.00}
\disputationslokal{Kollegiesalen, Administrationsbyggnaden,
  Kungl Tekniska h\"{o}gskolan, Valhallav\"{a}gen~79, Stockholm}
\isbn{ISBN x-xxxx-xxx-x}
\issn{ISSN xxxx-xxxx}
\isrn{ISRN KTH/xxx/xx-{}-yy/nn-{}-SE}
\trita{TRITA xxx yyyy-nn}
\publisher{Universitetsservice US AB}
\address{KTH School of Computer Science and Communication\\
  SE-100 44 Stockholm\\
  SWEDEN}
\kthlogo{kth_svv_comp_science_comm}
%</biblio>
%<*ex1|ex2>
\title{Lorem ipsum dolor sit amet, sed diam nonummy nibh eui
       mod tincidunt ut laoreet dol}
\subtitle{Duis autem vel eum iruire dolor in hendrerit in
          vulputate velit esse molestie consequat, vel illum
          dolore eu feugiat null}
\author{Namn Namnet}
\date{November 2003}
\blurb{Master's Thesis at NADA\\Supervisor: Tjoho\\Examiner: Tjohej}
\trita{TRITA xxx yyyy-nn}
%</ex1|ex2>
%    \end{macrocode}
%    (You can of course use the charcters \aa\ \"{a} and \"{o} in
%    your \LaTeX\ source file if you have specified the correct
%    input encoding with the inputenc package.)
%
%    Having defined all bibliographic information, it is
%    straightforward to typeset the start of the thesis.
%    \begin{macrocode}
%<*demo|ex1|ex2>
\begin{document}
\frontmatter
\maketitle
%%
%% This is file `kth-abs.tex',
%% generated with the docstrip utility.
%%
%% The original source files were:
%%
%% kthesis.dtx  (with options: `abstract')
%% 
%% IMPORTANT NOTICE:
%% 
%% For the copyright see the source file.
%% 
%% Any modified versions of this file must be renamed
%% with new filenames distinct from kth-abs.tex.
%% 
%% For distribution of the original source see the terms
%% for copying and modification in the file kthesis.dtx.
%% 
%% This generated file may be distributed as long as the
%% original source files, as listed above, are part of the
%% same distribution. (The sources need not necessarily be
%% in the same archive or directory.)
\begin{abstract}
  In this paper a scenario of one evader being chased  by multiple pursuers in two specific simulation environments is studied. The simulation environments are divided into an open area without obstacles and a closed area with obstacles. In the open
area a fairly accurate system of dynamics are implemented for both pursuers and evader. The Virtual Vehicle Approach is used to provide a reference trajectory for the pursuers to follow in order to catch the evader. The main purpose of this thesis is to find a decentralized robust
control method for the dynamics of the pursuers. In the closed area, the line of sight and field of view
are introduced and the solution to the Minimum time UGV surveillance problem and the Centroidal
Voronoi partitions. Different
capturing strategies, encirclement and one-on-one chase, are both studied and compared. The
numerical implementation and the resulting simulation are presented and analyzed. Conclusion on
the optimal formation for the multiple pursuers is made.
\end{abstract}
\endinput
%%
%% End of file `kth-abs.tex'.

\clearpage
%</demo|ex1|ex2>
%<*spik>
\begin{document}
\maketitle
%%
%% This is file `kth-abs.tex',
%% generated with the docstrip utility.
%%
%% The original source files were:
%%
%% kthesis.dtx  (with options: `abstract')
%% 
%% IMPORTANT NOTICE:
%% 
%% For the copyright see the source file.
%% 
%% Any modified versions of this file must be renamed
%% with new filenames distinct from kth-abs.tex.
%% 
%% For distribution of the original source see the terms
%% for copying and modification in the file kthesis.dtx.
%% 
%% This generated file may be distributed as long as the
%% original source files, as listed above, are part of the
%% same distribution. (The sources need not necessarily be
%% in the same archive or directory.)
\begin{abstract}
  In this paper a scenario of one evader being chased  by multiple pursuers in two specific simulation environments is studied. The simulation environments are divided into an open area without obstacles and a closed area with obstacles. In the open
area a fairly accurate system of dynamics are implemented for both pursuers and evader. The Virtual Vehicle Approach is used to provide a reference trajectory for the pursuers to follow in order to catch the evader. The main purpose of this thesis is to find a decentralized robust
control method for the dynamics of the pursuers. In the closed area, the line of sight and field of view
are introduced and the solution to the Minimum time UGV surveillance problem and the Centroidal
Voronoi partitions. Different
capturing strategies, encirclement and one-on-one chase, are both studied and compared. The
numerical implementation and the resulting simulation are presented and analyzed. Conclusion on
the optimal formation for the multiple pursuers is made.
\end{abstract}
\endinput
%%
%% End of file `kth-abs.tex'.

\end{document}
%</spik>
%    \end{macrocode}
%    Again, it is convenient to have the abstract in a separate
%    file so that it can be used both in the thesis itself and
%    on the announcement sheets.
%    \begin{macrocode}
%<*abstract>
\begin{abstract}
  This is a skeleton for KTH theses. More documentation
  regarding the KTH thesis class file can be found in
  the package documentation.
\end{abstract}
%</abstract>
%    \end{macrocode}
%    It is recommended that every book published in Sweden
%    contains a summary in Swedish.
%    \begin{macrocode}
%<*demo|ex1|ex2>
\selectlanguage{swedish}
\begin{abstract}
  Denna fil ger ett avhandlingsskelett.
  Mer information om \LaTeX-mallen finns i
  dokumentationen till paketet.
\end{abstract}
\selectlanguage{english}
\clearpage
%    \end{macrocode}
%    The remainder of the document is entered exactly as it
%    would be with the \class{memoir} class.
%    \begin{macrocode}
\tableofcontents
\mainmatter
%    \end{macrocode}
%    Now follows lots of nonsense text. We omit it in the manual\ldots
%
% \StopEventually
%
%    \begin{macrocode}
\chapter{Introduction}

Lorem ipsum dolor sit amet, consectetuer adipiscing elit. Mauris
purus. Fusce tempor. Nulla facilisi. Sed at turpis. Phasellus eu
ipsum. Nam porttitor laoreet nulla. Phasellus massa massa, auctor
rutrum, vehicula ut, porttitor a, massa. Pellentesque fringilla. Duis
nibh risus, venenatis ac, tempor sed, vestibulum at, tellus. Class
aptent taciti sociosqu ad litora torquent per conubia nostra, per
inceptos hymenaeos. Aliquam ligula lectus, volutpat in, rutrum sit
amet, ultrices at, elit. Aenean consequat facilisis sapien. Nulla
auctor, wisi vel semper ornare, augue augue facilisis turpis, a
laoreet quam libero quis erat. Vestibulum sagittis. Duis blandit
laoreet tortor. Nulla mattis. Morbi eget turpis. Praesent mi. Nunc
enim leo, blandit ut, sagittis at, pharetra sit amet, dui. Praesent
porta tortor a ligula rutrum dapibus.

Integer pellentesque, orci consequat suscipit pharetra, sem eros
tempor orci, non ultricies massa justo vitae nunc. Nunc sagittis.
Vestibulum egestas quam at eros. Curabitur consectetuer, magna vel
gravida varius, tortor augue congue tellus, quis gravida nulla risus
nec diam. Aliquam commodo dapibus risus. Vestibulum lacinia lectus sed
nisl. Sed libero purus, tincidunt non, sagittis nec, sodales quis,
nulla. Sed tortor tortor, viverra et, hendrerit eu, accumsan
sollicitudin, mi. Suspendisse potenti. Maecenas non wisi sed urna
mollis dignissim. Aenean at ligula. Vestibulum porta. Donec mauris.
Donec scelerisque rutrum elit. Aenean ultrices, eros vel sollicitudin
semper, dui lacus hendrerit metus, id ultricies arcu arcu id mi. In
eget sapien sed lorem cursus eleifend.

\part{Important Results}

\chapter{First One}

Aliquam et ante. Vivamus ultricies, neque eget iaculis interdum, lacus
quam hendrerit sapien, vel posuere justo nulla vitae arcu. Morbi
magna. Aliquam erat volutpat. Aenean mattis consequat nibh. Donec
lobortis sapien a enim. Cras mattis ultricies mi. Quisque venenatis.
Phasellus risus justo, vulputate non, tristique in, tristique vel,
neque. Phasellus pretium, dui nec dapibus laoreet, ligula enim laoreet
eros, tempus interdum massa turpis quis dolor. Etiam ultricies
condimentum neque. Maecenas pellentesque. Duis tortor. Aliquam ac
dolor. Vestibulum nisl. Nunc facilisis tincidunt mi. Morbi feugiat
velit vitae velit. Suspendisse potenti. Maecenas eget ante. Maecenas
blandit, urna at varius lacinia, lorem purus ullamcorper risus, non
pretium arcu libero at odio.

\section{Preliminaries}

Vestibulum dolor dolor, interdum eu, iaculis scelerisque, auctor at,
mi. Nullam ut erat quis sem bibendum dignissim. Vestibulum volutpat
nisl in tortor. Etiam pretium nulla et elit. Integer tortor. Ut metus
ligula, egestas non, ornare vel, bibendum eu, urna. Pellentesque
rutrum dui vel orci. Aliquam lacus lacus, varius a, sodales vitae,
egestas a, purus. Integer varius venenatis odio. Pellentesque dui
mauris, lacinia in, placerat lobortis, tempor ac, lectus.

\subsection{Remarks}

Sed lobortis neque non mauris. Maecenas rhoncus tempor justo. Nunc vel
diam at dolor luctus tincidunt. Ut egestas. Aliquam eu turpis eu nisl
ultricies sollicitudin. Donec erat odio, fermentum id, malesuada id,
viverra vitae, lorem. Morbi scelerisque sagittis enim. Donec non ante
sed neque dictum consectetuer. Etiam bibendum odio quis est. Quisque
mollis magna et odio. Nunc nulla. Suspendisse magna felis, tincidunt
in, blandit non, tempus vel, orci.

\subsection{Definitions}

Sed laoreet tellus in massa. Fusce egestas dapibus wisi. Pellentesque
sit amet nulla. Donec dignissim rutrum urna. Integer et mi at urna
viverra vestibulum. Nullam justo tortor, vulputate nec, porta non,
consequat ac, lectus. Sed ligula. Nullam tristique. Nullam pulvinar.
Morbi at elit et lacus tristique nonummy. Vivamus nunc ante, imperdiet
at, luctus non, aliquam a, neque. In interdum sapien et wisi.
Curabitur ipsum justo, consequat eget, viverra eu, porta vel, lacus.
Suspendisse potenti. Proin wisi. In tristique neque at ipsum.
Phasellus ac justo eu nibh faucibus lobortis. Vestibulum lorem.

Duis blandit est id diam. Sed aliquet semper arcu. Curabitur sed eros
ut neque eleifend sagittis. Cras lacinia, nisl a sagittis scelerisque,
nibh risus varius pede, ac interdum lacus lorem quis quam. Mauris sit
amet orci a ligula porttitor nonummy. Aliquam malesuada turpis at diam
semper vestibulum. Phasellus ullamcorper. Aenean ultrices lacus a diam
lobortis posuere. Duis porttitor euismod turpis. Ut sem nulla, porta
vel, lacinia eget, rutrum non, justo. Phasellus risus sem, placerat
at, sodales ac, viverra vitae, magna. Fusce non magna. Mauris varius
vestibulum eros. Nullam nec quam in tortor ultrices varius. Nulla
tempus lectus ut tortor.

Ut vestibulum diam in lacus. Aenean vestibulum bibendum dolor.
Praesent in wisi. Aenean ornare faucibus orci. Donec mattis magna et
dui. Nullam vestibulum lobortis nibh. Etiam ultrices lorem nec est.
Duis ut nisl. Vivamus risus. Pellentesque habitant morbi tristique
senectus et netus et malesuada fames ac turpis egestas. Suspendisse
sagittis, dolor eget ullamcorper elementum, sem erat sagittis dolor,
sit amet tincidunt quam neque consectetuer quam. Maecenas at quam.
Etiam gravida. Sed nec enim. Fusce pulvinar. In pede metus, lobortis
a, ullamcorper quis, commodo vitae, turpis. Maecenas sed mi sit amet
odio suscipit ornare. Phasellus pede arcu, elementum sed, elementum
eu, viverra et, ligula. Aliquam metus nisl, convallis et, ultrices
vel, consequat quis, felis. Aenean gravida euismod urna.

\section{The Main Theorem}

Aliquam quis nibh quis justo elementum viverra. Vestibulum ipsum.
Integer sit amet urna id lorem condimentum pretium. Nam adipiscing
lobortis purus. Donec at libero id augue interdum vulputate. Curabitur
imperdiet suscipit metus. Curabitur ac quam sed lacus accumsan
posuere. Quisque pharetra mi sit amet enim. Curabitur quis elit. Lorem
ipsum dolor sit amet, consectetuer adipiscing elit. Sed ultricies
aliquet lorem. Proin posuere tincidunt diam. Donec quis orci non leo
elementum nonummy. Donec urna lectus, fringilla at, tempus id, auctor
a, wisi. Pellentesque habitant morbi tristique senectus et netus et
malesuada fames ac turpis egestas. Suspendisse blandit. Nam wisi.
Phasellus egestas lacus ut lorem. Suspendisse sapien. Fusce non dolor
ac odio tempus placerat.

\subsection{Problem Statement}

Suspendisse id mauris a justo venenatis feugiat. Quisque volutpat,
ligula et fermentum volutpat, lectus magna tincidunt urna, a dignissim
augue sem eu ipsum. Aliquam vitae pede. Sed ullamcorper faucibus erat.
Aenean vehicula augue at dui. Donec adipiscing aliquet ligula. Sed sed
libero at enim ornare molestie. Suspendisse tristique sodales wisi. In
sit amet orci. In hac habitasse platea dictumst. Cras pede. Ut elit.

Morbi eu augue. Praesent in dui. Pellentesque habitant morbi tristique
senectus et netus et malesuada fames ac turpis egestas. Phasellus
felis. Morbi eleifend sagittis quam. Mauris imperdiet purus vel arcu.
Proin quis risus. Cum sociis natoque penatibus et magnis dis
parturient montes, nascetur ridiculus mus. Mauris massa justo, iaculis
nec, commodo suscipit, faucibus at, nunc. Phasellus libero risus,
ultrices eu, congue eu, sollicitudin et, tortor. Integer ante turpis,
malesuada vel, sagittis nec, porta ut, dolor. Integer ultricies
blandit odio. Praesent et leo sit amet ipsum nonummy sodales.

\subsection{The Proof}

Suspendisse aliquam fringilla tortor. Nullam lacinia rutrum orci. Cras
pellentesque iaculis ligula. Fusce elit enim, nonummy ut, fermentum
sed, nonummy a, nunc. Suspendisse potenti. Nunc tortor dolor, eleifend
non, aliquam sit amet, vestibulum id, erat. Curabitur imperdiet.
Quisque in velit. Etiam a urna sed turpis scelerisque tempor. Praesent
in dolor eget massa congue aliquet. Praesent magna lacus, dictum a,
accumsan a, sollicitudin vel, nibh. Phasellus accumsan porta tortor.
Curabitur mauris. Quisque ut wisi. Sed aliquet molestie mi. Quisque ut
magna. Vivamus rhoncus urna in libero.

Nullam tristique tempus neque. Nunc ac tortor lobortis felis nonummy
lacinia. Pellentesque vestibulum facilisis quam. Aliquam aliquam,
lorem eget elementum euismod, urna dui rutrum libero, a imperdiet erat
augue ut ipsum. Morbi gravida turpis ut sapien. Aliquam ante sapien,
ultricies at, vulputate non, ullamcorper quis, dolor. Quisque velit
turpis, consectetuer vitae, facilisis feugiat, volutpat eget, nunc.
Quisque quis ipsum in diam elementum vulputate. Donec convallis sapien
ut tellus. Cras dictum turpis nec sem. Nunc augue. Aliquam nunc
tortor, semper vel, aliquam at, tempus ut, risus. Mauris feugiat nunc
id justo. Nunc ullamcorper magna vel urna. Morbi rutrum massa non
sapien. Fusce purus. Cras magna. In condimentum augue ac libero. Nulla
lacinia vulputate leo. Mauris lobortis molestie turpis.

Pellentesque neque odio, ornare a, hendrerit ac, vestibulum sit amet,
tortor. Vestibulum ante ipsum primis in faucibus orci luctus et
ultrices posuere cubilia Curae; Phasellus malesuada porta sem. Nunc
wisi neque, lobortis at, consequat sit amet, pulvinar eget, mauris.
Aliquam egestas, arcu ut tristique tempus, urna lorem pulvinar nisl,
at auctor lectus sem ut felis. Duis hendrerit, mi et varius venenatis,
nibh libero blandit mauris, bibendum pellentesque sapien dolor quis
velit. Vivamus nec augue ut velit eleifend accumsan. Fusce euismod.
Phasellus volutpat. Donec sollicitudin. Donec non mi eu lorem suscipit
tincidunt. Suspendisse vehicula. Integer iaculis diam ac quam.
Pellentesque velit mi, pulvinar eget, suscipit sed, rhoncus ut,
tortor. Etiam ultricies, wisi non ornare rutrum, ipsum nunc posuere
leo, vitae blandit est enim eu massa. Nulla sed felis vel nunc
eleifend venenatis. Sed non purus. Aliquam enim. Nulla neque massa,
elementum sit amet, rhoncus vel, rhoncus in, leo.

In convallis pellentesque quam. Fusce venenatis neque id justo.
Aliquam felis. Nullam vitae eros. Sed sit amet lorem. Vestibulum
elementum purus in nunc. Phasellus nec libero non ipsum ultrices
porttitor. Vestibulum ante ipsum primis in faucibus orci luctus et
ultrices posuere cubilia Curae; Praesent quis elit sed eros bibendum
tempor. Suspendisse tempor congue nisl. Nam a est. Aliquam erat
volutpat. Quisque tincidunt. Donec egestas dapibus diam. Aenean ante
mi, vulputate vitae, adipiscing in, fermentum luctus, wisi. In nibh
mauris, ultricies a, fermentum eget, faucibus pharetra, neque. Cras
consectetuer congue ipsum. Nulla eu metus. Sed at turpis.

\chapter{Second One}

Vestibulum sollicitudin magna quis ipsum. Phasellus nec dolor. Quisque
laoreet, mi et porttitor tincidunt, leo mauris ultrices purus, eget
tempus orci neque vel sapien. Pellentesque tempus pulvinar diam. Ut
eleifend posuere lectus. Maecenas at justo vel lorem auctor
pellentesque. Cras mattis laoreet ante. Mauris vitae sem et tellus
gravida ornare. Donec a risus. Pellentesque ac nunc at leo tempus
sagittis. Morbi enim. Sed nibh quam, fermentum id, hendrerit id,
semper et, metus. Nullam congue enim id eros. Phasellus facilisis
mollis mi. Nam laoreet cursus justo. Curabitur placerat varius lectus.

\section{Preliminaries}

Donec viverra, velit non dapibus consectetuer, est tellus porta orci,
vel convallis elit tortor sed massa. Aliquam erat volutpat. Nullam et
nibh. Nunc posuere. Nullam vulputate convallis ligula. Integer lorem
wisi, porta eget, placerat quis, sollicitudin vitae, diam. Maecenas
non tellus in tellus mollis placerat. Aenean eget ligula. Suspendisse
bibendum eros non orci. Vestibulum ante ipsum primis in faucibus orci
luctus et ultrices posuere cubilia Curae; Quisque id tellus ac est
suscipit sodales. Curabitur placerat, odio ut cursus accumsan, libero
lectus consectetuer risus, at lobortis ligula nibh quis odio.
Phasellus ac nibh. Vivamus a lectus ut pede scelerisque aliquam. Sed
vestibulum erat vitae nunc. Quisque non neque.

Aliquam augue. Mauris ipsum wisi, luctus et, volutpat non, tincidunt
ac, augue. Morbi posuere diam nec arcu. In ut mi sit amet turpis
fermentum posuere. Maecenas in lorem. Maecenas tincidunt accumsan
odio. Nulla mauris. Cras ac elit eget tortor vulputate venenatis. Nam
quam urna, consequat nec, dictum sed, mollis nec, justo. Ut velit.
Nunc nibh arcu, consequat at, venenatis eu, ultrices cursus, lectus.
Sed blandit blandit ante. Praesent in nisl. Nullam nibh erat, ornare
quis, iaculis in, imperdiet id, ligula. Nullam odio velit, semper ac,
lobortis sed, rhoncus vitae, dolor. Duis ultricies libero sed velit.
Cum sociis natoque penatibus et magnis dis parturient montes, nascetur
ridiculus mus. Morbi vestibulum magna a purus.

\subsection{Remarks}

In hac habitasse platea dictumst. Suspendisse sit amet dolor a wisi
auctor interdum. Duis ut diam ut arcu sodales pulvinar. Etiam a
ligula. Proin arcu. Fusce pulvinar porta felis. Vestibulum vulputate.
Nullam posuere, nunc ac rhoncus auctor, dolor risus vestibulum nibh,
non fringilla nunc nibh quis lacus. Suspendisse scelerisque
pellentesque felis. Suspendisse nisl nibh, viverra sit amet, lobortis
in, commodo ac, arcu. Aenean blandit erat nec magna. Praesent
fermentum.

\subsection{Definitions}

Sed consectetuer vestibulum risus. Fusce eget ligula. Maecenas felis
dui, posuere a, accumsan quis, fringilla ac, est. Morbi vestibulum,
orci vel luctus cursus, lectus dui interdum wisi, nec posuere metus
nibh vitae sapien. Praesent quis mi at libero volutpat vehicula. Nulla
facilisi. Vivamus dapibus venenatis eros. Integer tristique lacinia
mauris. Ut vulputate faucibus mi. Morbi pretium pellentesque mi.
Aliquam in sapien sed sapien ultricies feugiat. Sed et est a velit
lacinia convallis. Etiam velit nisl, vehicula nec, sagittis et, porta
sed, sapien. In dignissim. Sed ligula mauris, lobortis eget,
scelerisque id, vestibulum et, lorem. Mauris consequat metus nec
risus. Nulla auctor nulla id dolor.

In hac habitasse platea dictumst. Proin aliquam tincidunt risus.
Mauris ipsum. Suspendisse lectus. Quisque at augue. Praesent
adipiscing tortor vel dolor. Pellentesque habitant morbi tristique
senectus et netus et malesuada fames ac turpis egestas. Nulla neque.
Sed at orci. Fusce vestibulum aliquam libero. Nam adipiscing, metus
sit amet aliquam pulvinar, magna nunc tristique pede, pellentesque
porta erat eros eget odio. Aenean turpis tortor, porta vel, vestibulum
a, imperdiet eget, wisi. Curabitur massa risus, consectetuer at,
hendrerit eu, convallis non, libero. Etiam nulla tortor, pharetra ut,
bibendum et, molestie nec, sapien. Nullam sollicitudin. Fusce sit amet
nunc nec ipsum vehicula bibendum. Maecenas eu lacus. Nulla interdum,
neque sit amet pharetra interdum, erat pede eleifend lectus, ac
consectetuer purus urna quis wisi. Integer pretium nunc ut dui.

\section{The Main Theorem}

Donec ultricies. Proin odio. Ut ac turpis at justo tempus luctus.
Quisque mi. Integer eu velit. Proin pharetra, wisi non dignissim
tristique, dui felis pharetra metus, in luctus leo nulla nec libero.
Duis porttitor. Praesent pellentesque vulputate metus. Duis auctor
sagittis velit. Pellentesque mollis, dui et convallis sagittis, lectus
felis pulvinar arcu, at lacinia erat lacus in tortor. Pellentesque
wisi neque, consequat in, dignissim quis, pretium quis, massa.

Donec bibendum vestibulum wisi. Nam facilisis. Phasellus dignissim
tempor quam. Vestibulum placerat venenatis erat. Sed wisi felis,
interdum eget, porttitor varius, auctor ac, felis. Sed at sem. Ut eget
nisl. Aliquam eu arcu eget nunc hendrerit vulputate. Vivamus eu
tortor. Lorem ipsum dolor sit amet, consectetuer adipiscing elit.

Vivamus id lacus. In tempus velit. Vestibulum ante ipsum primis in
faucibus orci luctus et ultrices posuere cubilia Curae; Aenean wisi
lacus, eleifend id, vehicula et, malesuada vitae, wisi. Phasellus
tristique enim eget tellus. Cras mauris libero, consequat et, dictum
non, lobortis id, urna. Maecenas venenatis. Etiam rutrum, purus nec
posuere facilisis, erat libero condimentum magna, nec congue leo pede
sit amet ante. Quisque dictum, nunc ut dignissim convallis, neque
metus luctus mi, vel euismod leo metus a augue. Sed felis. In laoreet,
pede sit amet vestibulum nonummy, nibh lacus lacinia dolor, et
placerat sem erat vitae metus. Donec pellentesque placerat nisl.
Vestibulum eget nunc. Cras congue dolor sed ante. Donec nisl velit,
commodo quis, porta sed, vestibulum in, ante. Donec ante. Morbi
fermentum, eros non mattis fermentum, ipsum mi tempor tellus, eu
imperdiet odio ante id mauris. Integer tempus, sem vitae mollis
tempor, ante nulla condimentum leo, vitae mattis lectus purus vitae
risus. Suspendisse potenti.

\subsection{Problem Statement}

Vestibulum ante ipsum primis in faucibus orci luctus et ultrices
posuere cubilia Curae; Nunc vel sem in diam fringilla commodo. Cum
sociis natoque penatibus et magnis dis parturient montes, nascetur
ridiculus mus. Vivamus sed mi. Praesent enim augue, vulputate ac,
interdum id, feugiat eget, justo. Quisque elit. Pellentesque habitant
morbi tristique senectus et netus et malesuada fames ac turpis
egestas. Nam et purus. Etiam non orci vel nibh euismod viverra.
Pellentesque eu velit. Fusce vestibulum, ipsum non fringilla
consectetuer, turpis dui adipiscing lorem, sed adipiscing tortor eros
in libero. Sed congue sem et wisi.

Proin mattis, enim vel condimentum sollicitudin, justo nunc porttitor
felis, in mattis wisi odio id ipsum. Vivamus lectus. Etiam blandit
elementum wisi. Quisque sit amet est convallis leo faucibus tincidunt.
Morbi lorem arcu, nonummy in, ultrices vel, vulputate nec, urna. Nulla
facilisi. Suspendisse vitae purus. Maecenas imperdiet, eros sit amet
volutpat dapibus, nisl eros bibendum libero, vel laoreet massa ipsum
eu enim. Aliquam in elit. Cum sociis natoque penatibus et magnis dis
parturient montes, nascetur ridiculus mus. Donec eu elit. Vestibulum
sit amet lacus quis nulla laoreet aliquam. In pharetra lectus eu dui.

\subsection{The Proof}

Curabitur molestie, arcu vitae varius rhoncus, nunc erat feugiat
tellus, ut volutpat massa ipsum sed quam. Nullam mauris massa, mattis
eget, sagittis at, tincidunt a, leo. Donec placerat. Proin lorem mi,
semper at, imperdiet vel, fermentum et, eros. Curabitur dui velit,
blandit fringilla, consequat quis, sagittis ut, lectus. Fusce non
risus eget sem fermentum feugiat. Nullam justo nunc, faucibus et,
pharetra nec, pellentesque sit amet, tellus. Cras fringilla lectus vel
augue. Donec sapien. Aliquam malesuada, ligula vel venenatis cursus,
ante metus interdum quam, in tristique risus pede non turpis. Sed
dignissim. Aliquam erat volutpat. Aenean porttitor wisi a ante. Nulla
leo nibh, bibendum malesuada, suscipit nec, consequat ut, justo.

Morbi sit amet nunc et neque faucibus adipiscing. Donec lacinia purus
ut metus. Maecenas sodales justo quis wisi. Aliquam id lacus ac nisl
rutrum porttitor. Fusce ac neque. Nulla facilisi. Vestibulum dignissim
iaculis lorem. Vivamus scelerisque ante ac nibh. Sed non sapien. Donec
egestas pharetra sem. Suspendisse tellus lorem, gravida eu, pulvinar
a, ornare vitae, urna. Curabitur justo. Fusce ante enim, fermentum
eget, bibendum eu, pulvinar id, felis. Vestibulum ante ipsum primis in
faucibus orci luctus et ultrices posuere cubilia Curae; Ut vestibulum
lacus nec libero. Sed tortor.

Aenean vel risus quis tortor vehicula dictum. Aliquam nec sapien.
Class aptent taciti sociosqu ad litora torquent per conubia nostra,
per inceptos hymenaeos. Quisque sagittis suscipit mauris. Proin
vestibulum leo id purus. Mauris lobortis. Proin eleifend. Aliquam eros
dolor, semper nec, sagittis vel, sollicitudin at, felis. Quisque
blandit mattis ante. Maecenas leo. Praesent libero. Donec orci ipsum,
commodo vitae, sagittis quis, vehicula vel, nulla. Praesent non arcu a
mauris porta euismod. Pellentesque in sem.

Aliquam ut urna. Nulla in libero quis libero fermentum semper. Lorem
ipsum dolor sit amet, consectetuer adipiscing elit. Aenean ac quam a
odio tristique lacinia. Lorem ipsum dolor sit amet, consectetuer
adipiscing elit. Vivamus quis sem. Donec risus risus, viverra vitae,
ornare et, accumsan id, dolor. Sed wisi tortor, viverra non, ultricies
in, pulvinar vel, elit. Class aptent taciti sociosqu ad litora
torquent per conubia nostra, per inceptos hymenaeos. Vivamus ultrices
accumsan purus. In in risus. Praesent condimentum, pede a egestas
placerat, nulla leo mollis wisi, in mattis elit orci quis justo.
Aliquam erat volutpat. Nunc augue. Morbi suscipit posuere dolor.
Vivamus aliquet lorem sed enim. Fusce a est ac pede congue pharetra.

Cras odio orci, consectetuer a, porttitor eu, scelerisque eu, quam.
Vivamus imperdiet tellus. Aenean sed leo. Aenean molestie vestibulum
pede. Aenean eleifend dolor vitae magna. Maecenas malesuada, nunc
iaculis ornare tempus, lacus velit pharetra sem, id fringilla eros dui
eget arcu. Integer eget arcu. Morbi porta. Nam justo turpis,
adipiscing sed, euismod sed, auctor non, nulla. Duis velit massa,
congue sit amet, sagittis at, consequat id, risus. Nulla purus diam,
nonummy ut, euismod ac, pretium et, leo. In at est. Phasellus
malesuada libero sed tellus. Integer mattis urna ut orci viverra
ultrices. Etiam varius, sapien vel mattis malesuada, felis est blandit
odio, sed lobortis sapien erat non ipsum. Nunc malesuada, felis quis
faucibus imperdiet, enim purus posuere quam, nec lobortis lacus mauris
vitae lacus. Aliquam erat volutpat. Sed eleifend convallis neque.
Curabitur in mauris et magna varius lobortis.

Fusce vel nunc. Mauris sagittis ornare wisi. Suspendisse urna. Nunc
euismod purus et risus. Praesent at quam ut ante dignissim dapibus.
Mauris id leo. Pellentesque nec quam. Donec eget eros. Donec elit
erat, placerat in, luctus nec, varius ac, felis. Donec sollicitudin
mauris eget lacus. Cras odio lacus, pharetra at, vestibulum non,
aliquet sed, augue. Morbi mollis diam sed quam. Nam nec arcu in turpis
ultricies venenatis.

\part{Less Important Results}

\chapter{Last One}

Ut iaculis fermentum augue. Maecenas vel pede. Nunc enim odio, commodo
vitae, faucibus vel, ultricies vitae, mauris. Duis vitae nibh a nunc
consequat bibendum. Nam tincidunt libero sodales dolor. Aliquam
tempus, mi vitae ultrices facilisis, nibh lacus cursus wisi, at
tristique mi odio eget nulla. Donec lectus risus, mollis bibendum,
suscipit nec, aliquam quis, eros. Donec tortor. Duis blandit
condimentum orci. Quisque in ante vitae risus rutrum feugiat. Nulla
sapien. Proin ipsum nibh, condimentum at, ullamcorper vel, bibendum
et, elit. Etiam malesuada eros sit amet pede. Duis arcu. Fusce a mi.

Donec in elit. Integer leo. Curabitur ornare volutpat erat. Vestibulum
lacinia, eros sit amet dictum consectetuer, neque eros auctor elit, et
mollis felis metus a ligula. In lacinia sapien sed dolor. Maecenas non
magna id wisi dapibus dignissim. Suspendisse fringilla diam at lacus.
Praesent sit amet lorem. Vivamus fermentum, felis at vehicula semper,
leo elit pulvinar neque, quis posuere ipsum elit eu odio. Donec et dui
non quam sodales interdum.

\section{Definitions}

Aenean posuere velit eget magna. Sed neque tellus, tempor eget, mollis
eget, pulvinar sed, elit. Phasellus feugiat mattis tortor. Nulla sed
massa quis lacus tincidunt hendrerit. Praesent aliquet tristique arcu.
Vestibulum sed est at odio blandit porttitor. Quisque a justo. Nunc a
libero. Curabitur ut libero quis orci tempor interdum. Quisque luctus
pede vel mauris. Donec velit. Vestibulum placerat pulvinar elit.
Aliquam erat volutpat.

Fusce tristique enim eget est. Donec vulputate, tortor vel dapibus
pretium, pede ante sagittis ligula, eu tempus lorem eros sit amet
elit. Pellentesque pede pede, sodales id, nonummy eget, condimentum
ut, urna. Quisque sed metus nec lorem imperdiet eleifend. Phasellus
non massa vel leo pretium tristique. Ut feugiat accumsan magna. Nulla
lectus risus, ultricies id, suscipit ac, commodo sed, nulla. Donec
neque eros, sollicitudin sit amet, mollis sit amet, bibendum ac, sem.
Aliquam sed nisl id diam vestibulum accumsan. Integer purus quam,
commodo id, placerat ut, malesuada id, magna. Suspendisse elit turpis,
vehicula eu, imperdiet et, aliquet ac, risus. Nulla mollis nibh sit
amet magna. Nullam dui pede, volutpat in, tempus eu, rhoncus eget,
orci. Pellentesque in neque. Nunc eu pede ut sem tempus dictum. Donec
vel mauris placerat lorem consectetuer sollicitudin. Praesent ultrices
lobortis sapien. Mauris vel nisl vitae elit ullamcorper dapibus.

\section{The Main Theorem}

Sed tincidunt erat quis leo. Integer vestibulum massa a tellus.
Pellentesque eu lacus in quam euismod volutpat. Donec ut nulla. Morbi
mi diam, posuere ac, consectetuer in, iaculis quis, ante. Ut eleifend.
Morbi sagittis, diam et rutrum molestie, orci magna tempus lectus, non
euismod nunc quam nec velit. Donec leo. Cum sociis natoque penatibus
et magnis dis parturient montes, nascetur ridiculus mus. Quisque
hendrerit. Morbi mattis. Vivamus vulputate dolor in ante. Curabitur
accumsan adipiscing leo. In quis ante. Morbi arcu. Curabitur varius
rutrum sem.


Donec facilisis, lacus sollicitudin malesuada pharetra, eros massa
elementum massa, eget consequat felis leo et libero. Vivamus id diam
quis ligula vehicula varius. Nunc tempus massa vel orci. Fusce
laoreet. Praesent et risus. Vestibulum ante ipsum primis in faucibus
orci luctus et ultrices posuere cubilia Curae; Fusce metus elit,
fringilla nec, facilisis id, sollicitudin in, tortor. Curabitur tellus
ante, lacinia eget, gravida nec, sollicitudin quis, quam. Integer
vitae felis. Nullam tellus purus, feugiat vitae, ultricies ut, feugiat
eu, turpis. Praesent urna orci, aliquet a, fringilla vitae, molestie
ac, pede. Vivamus quam orci, luctus nec, aliquet eu, consectetuer ut,
magna.

\subsection{The Proof}

Sed venenatis feugiat arcu. Nullam neque. Aliquam erat volutpat.
Praesent euismod, nibh ut interdum tincidunt, leo metus facilisis
ligula, ut sagittis est elit non elit. Nulla vitae augue. Cras id
diam. Maecenas vel tortor ac dolor porttitor cursus. Phasellus metus
ipsum, condimentum a, mollis eget, dapibus ac, lorem. Nullam sodales
vulputate sapien. Praesent rutrum blandit augue. Etiam in pede in
neque luctus hendrerit. Vivamus eget ante. Mauris turpis ipsum,
egestas in, commodo eget, dapibus et, dui. Donec scelerisque. Sed a
neque. Integer at quam eu augue imperdiet vehicula. Aliquam orci eros,
pellentesque eget, adipiscing eu, tristique ut, tellus.

Aenean quis justo at mi vestibulum bibendum. Sed at pede eu lacus
euismod condimentum. Cras in lacus. Nulla facilisi. Donec eu arcu quis
wisi fringilla venenatis. Pellentesque habitant morbi tristique
senectus et netus et malesuada fames ac turpis egestas. Quisque neque.
Sed scelerisque rhoncus est. Maecenas a dolor. Aliquam non justo.
Pellentesque habitant morbi tristique senectus et netus et malesuada
fames ac turpis egestas.

\end{document}
%</demo|ex1|ex2>
%    \end{macrocode}
%
% \section{The Bibliography Styles}
%
%    \begin{macrocode}
%<*bstfile>
ENTRY
  { address
    author
    booktitle
    chapter
    edition
    editor
    howpublished
    institution
    isbn
    issn
    journal
    key
    month
    note
    number
    organization
    pages
    publisher
    school
    series
    title
    type
    url
    volume
    year
  }
  {}
%<nat>  { label extra.label sort.label short.list }
%<plain>  { label }

INTEGERS { output.state before.all mid.sentence after.sentence after.block }

FUNCTION {init.state.consts}
{ #0 'before.all :=
  #1 'mid.sentence :=
  #2 'after.sentence :=
  #3 'after.block :=
}

STRINGS { s t }

FUNCTION {output.nonnull}
{ 's :=
  output.state mid.sentence =
    { ", " * write$ }
    { output.state after.block =
        { add.period$ write$
          newline$
          "\newblock " write$
        }
        { output.state before.all =
            'write$
            { add.period$ " " * write$ }
          if$
        }
      if$
      mid.sentence 'output.state :=
    }
  if$
  s
}

FUNCTION {output}
{ duplicate$ empty$
    'pop$
    'output.nonnull
  if$
}

FUNCTION {output.check}
{ 't :=
  duplicate$ empty$
    { pop$ "empty " t * " in " * cite$ * warning$ }
    'output.nonnull
  if$
}

FUNCTION {fin.entry}
{ add.period$
  write$
  newline$
}

FUNCTION {new.block}
{ output.state before.all =
    'skip$
    { after.block 'output.state := }
  if$
}

FUNCTION {new.sentence}
{ output.state after.block =
    'skip$
    { output.state before.all =
        'skip$
        { after.sentence 'output.state := }
      if$
    }
  if$
}

FUNCTION {not}
{   { #0 }
    { #1 }
  if$
}

FUNCTION {and}
{   'skip$
    { pop$ #0 }
  if$
}

FUNCTION {or}
{   { pop$ #1 }
    'skip$
  if$
}

FUNCTION {new.block.checka}
{ empty$
    'skip$
    'new.block
  if$
}

FUNCTION {new.block.checkb}
{ empty$
  swap$ empty$
  and
    'skip$
    'new.block
  if$
}

FUNCTION {new.sentence.checka}
{ empty$
    'skip$
    'new.sentence
  if$
}

FUNCTION {new.sentence.checkb}
{ empty$
  swap$ empty$
  and
    'skip$
    'new.sentence
  if$
}

FUNCTION {field.or.null}
{ duplicate$ empty$
    { pop$ "" }
    'skip$
  if$
}

FUNCTION {emphasize}
{ duplicate$ empty$
    { pop$ "" }
    { "\emph{" swap$ * "}" * }
  if$
}

FUNCTION {etal}
{
   " \textit{et al.}"
}

FUNCTION {et}
{
   " and "
}

INTEGERS { nameptr namesleft numnames }

FUNCTION {format.names}
{ 's :=
  #1 'nameptr :=
  s num.names$ 'numnames :=
  numnames 'namesleft :=
    { namesleft #0 > }
    { s nameptr "{ff~}{vv~}{ll}{, jj}" format.name$ 't :=
      nameptr #1 >
        { namesleft #1 >
            { ", " * t * }
            { numnames #2 >
                { "," * }
                'skip$
              if$
              t "others" =
                { etal * }
                { et * t * }
              if$
            }
          if$
        }
        't
      if$
      nameptr #1 + 'nameptr :=
      namesleft #1 - 'namesleft :=
    }
  while$
}

FUNCTION {format.authors}
{ author empty$
    { "" }
    { author format.names }
  if$
}

FUNCTION {format.editors}
{ editor empty$
    { "" }
    { editor format.names
      editor num.names$ #1 >
        { ", editors" * }
        { ", editor" * }
      if$
    }
  if$
}

FUNCTION {format.isbn}
{ isbn empty$
    { "" }
    { new.block "ISBN " isbn * }
  if$
}

FUNCTION {format.issn}
{ issn empty$
    { "" }
    { new.block "ISSN " issn * }
  if$
}

FUNCTION {format.url}
{ url empty$
    { "" }
    { new.block "URL \url{" url * "}" * }
  if$
}

FUNCTION {format.title}
{ title empty$
    { "" }
    { title "t" change.case$ }
  if$
}

%<*nat>
FUNCTION {format.key}
{ empty$
    { key field.or.null }
    { "" }
  if$
}

FUNCTION {format.full.names}
{'s :=
  #1 'nameptr :=
  s num.names$ 'numnames :=
  numnames 'namesleft :=
    { namesleft #0 > }
    { s nameptr
      "{vv~}{ll}" format.name$ 't :=
      nameptr #1 >
        {
          namesleft #1 >
            { ", " * t * }
            {
              numnames #2 >
                { "," * }
                'skip$
              if$
              t "others" =
                { etal * }
                { et * t * }
              if$
            }
          if$
        }
        't
      if$
      nameptr #1 + 'nameptr :=
      namesleft #1 - 'namesleft :=
    }
  while$
}

FUNCTION {author.editor.full}
{ author empty$
    { editor empty$
        { "" }
        { editor format.full.names }
      if$
    }
    { author format.full.names }
  if$
}

FUNCTION {author.full}
{ author empty$
    { "" }
    { author format.full.names }
  if$
}

FUNCTION {editor.full}
{ editor empty$
    { "" }
    { editor format.full.names }
  if$
}

FUNCTION {make.full.names}
{ type$ "book" =
  type$ "inbook" =
  or
    'author.editor.full
    { type$ "proceedings" =
        'editor.full
        'author.full
      if$
    }
  if$
}
%</nat>

FUNCTION {output.bibitem}
{ newline$
%<*nat>
  "\bibitem[" write$
  label write$
  ")" make.full.names duplicate$ short.list =
     { pop$ }
     { * }
   if$
  "]{" * write$
%</nat>
%<*plain>
  "\bibitem{" write$
%</plain>
  cite$ write$
  "}" write$
  newline$
  ""
  before.all 'output.state :=
}

FUNCTION {n.dashify}
{ 't :=
  ""
    { t empty$ not }
    { t #1 #1 substring$ "-" =
        { t #1 #2 substring$ "--" = not
            { "--" *
              t #2 global.max$ substring$ 't :=
            }
            {   { t #1 #1 substring$ "-" = }
                { "-" *
                  t #2 global.max$ substring$ 't :=
                }
              while$
            }
          if$
        }
        { t #1 #1 substring$ *
          t #2 global.max$ substring$ 't :=
        }
      if$
    }
  while$
}

%<*nat>
FUNCTION {format.date}
{
  month empty$
    { "" }
    {
      year empty$
        { "" }
        { year }
      if$
      month
      " " * swap$ *
    }
  if$
}

FUNCTION {format.year}
{ year duplicate$ empty$
    { "empty year in " cite$ * warning$
       pop$ "" }
    'skip$
  if$
  extra.label *
}
%</nat>
%<*plain>
FUNCTION {format.date}
{ year empty$
    { month empty$
        { "" }
        { "there's a month but no year in " cite$ * warning$
          month
        }
      if$
    }
    { month empty$
        'year
        { month " " * year * }
      if$
    }
  if$
}
%</plain>

FUNCTION {format.btitle}
{ title emphasize
}

FUNCTION {tie.or.space.connect}
{ duplicate$ text.length$ #3 <
    { "~" }
    { " " }
  if$
  swap$ * *
}

FUNCTION {either.or.check}
{ empty$
    'pop$
    { "can't use both " swap$ * " fields in " * cite$ * warning$ }
  if$
}

FUNCTION {format.bvolume}
{ volume empty$
    { "" }
    { "volume" volume tie.or.space.connect
      series empty$
        'skip$
        { " of " * series emphasize * }
      if$
      "volume and number" number either.or.check
    }
  if$
}

FUNCTION {format.number.series}
{ volume empty$
    { number empty$
        { series field.or.null }
        { output.state mid.sentence =
            { "number" }
            { "Number" }
          if$
          number tie.or.space.connect
          series empty$
            { "there's a number but no series in " cite$ * warning$ }
            { " in " * series * }
          if$
        }
      if$
    }
    { "" }
  if$
}

FUNCTION {format.edition}
{ edition empty$
    { "" }
    { output.state mid.sentence =
        { edition "l" change.case$ " edition" * }
        { edition "t" change.case$ " edition" * }
      if$
    }
  if$
}

INTEGERS { multiresult }

FUNCTION {multi.page.check}
{ 't :=
  #0 'multiresult :=
    { multiresult not
      t empty$ not
      and
    }
    { t #1 #1 substring$
      duplicate$ "-" =
      swap$ duplicate$ "," =
      swap$ "+" =
      or or
        { #1 'multiresult := }
        { t #2 global.max$ substring$ 't := }
      if$
    }
  while$
  multiresult
}

FUNCTION {format.pages}
{ pages empty$
    { "" }
    { pages multi.page.check
        { "pages" pages n.dashify tie.or.space.connect }
        { "page" pages tie.or.space.connect }
      if$
    }
  if$
}

FUNCTION {format.vol.num.pages}
{ volume field.or.null
  number empty$
    'skip$
    { "\penalty0 (" number * ")" * *
      volume empty$
        { "there's a number but no volume in " cite$ * warning$ }
        'skip$
      if$
    }
  if$
  pages empty$
    'skip$
    { duplicate$ empty$
        { pop$ format.pages }
        { ":\penalty0 " * pages n.dashify * }
      if$
    }
  if$
}

FUNCTION {format.chapter.pages}
{ chapter empty$
    'format.pages
    { type empty$
        { "chapter" }
        { type "l" change.case$ }
      if$
      chapter tie.or.space.connect
      pages empty$
        'skip$
        { ", " * format.pages * }
      if$
    }
  if$
}

FUNCTION {format.in.ed.booktitle}
{ booktitle empty$
    { "" }
    { editor empty$
        { "In " booktitle emphasize * }
        { "In " format.editors * ", " * booktitle emphasize * }
      if$
    }
  if$
}

FUNCTION {empty.misc.check}
{ author empty$ title empty$ howpublished empty$
  month empty$ year empty$ note empty$
  and and and and and
  key empty$ not and
    { "all relevant fields are empty in " cite$ * warning$ }
    'skip$
  if$
}

FUNCTION {format.thesis.type}
{ type empty$
    'skip$
    { pop$
      type "t" change.case$
    }
  if$
}

FUNCTION {format.tr.number}
{ type empty$
    { "Technical Report" }
    'type
  if$
  number empty$
    { "t" change.case$ }
    { number tie.or.space.connect }
  if$
}

FUNCTION {format.article.crossref}
{ key empty$
    { journal empty$
        { "need key or journal for " cite$ * " to crossref " * crossref *
          warning$
          ""
        }
        { "In " journal emphasize * }
      if$
    }
    { "In " key * }
  if$
%<nat>  " \citep{" * crossref * "}" *
%<plain>  " \cite{" * crossref * "}" *
}

%<*plain>
FUNCTION {format.crossref.editor}
{ editor #1 "{vv~}{ll}" format.name$
  editor num.names$ duplicate$
  #2 >
    { pop$ etal * }
    { #2 <
        'skip$
        { editor #2 "{ff }{vv }{ll}{ jj}" format.name$ "others" =
            { etal * }
            { et * editor #2 "{vv~}{ll}" format.name$ * }
          if$
        }
      if$
    }
  if$
}
%</plain>

FUNCTION {format.book.crossref}
{ volume empty$
    { "empty volume in " cite$ * "'s crossref of " * crossref * warning$
      "In "
    }
    { "Volume" volume tie.or.space.connect
      " of " *
    }
  if$
  editor empty$
  editor field.or.null author field.or.null =
  or
    { key empty$
        { series empty$
            { "need editor, key, or series for " cite$ * " to crossref " *
              crossref * warning$
              "" *
            }
            { series emphasize * }
          if$
        }
        { key * }
      if$
    }
%<nat>    'skip$
%<plain>    { format.crossref.editor * }
  if$
%<nat>  ", \citet{" * crossref * "}" *
%<plain>  " \cite{" * crossref * "}" *
}

FUNCTION {format.incoll.inproc.crossref}
{ editor empty$
  editor field.or.null author field.or.null =
  or
    { key empty$
        { booktitle empty$
            { "need editor, key, or booktitle for " cite$ * " to crossref " *
              crossref * warning$
              ""
            }
            { "In " booktitle emphasize * }
          if$
        }
        { "In " key * }
      if$
    }
%<nat>    { "In " }
%<plain>    { "In " format.crossref.editor * }
  if$
%<nat>  " \citet{" * crossref * "}" *
%<plain>  " \cite{" * crossref * "}" *
}

FUNCTION {article}
{ output.bibitem
  format.authors "author" output.check
%<*nat>
  author format.key output
  new.sentence
  format.year "year" output.check
%</nat>
  new.block
  format.title "title" output.check
  new.block
  crossref missing$
    { journal emphasize "journal" output.check
      format.vol.num.pages output
%<plain>      format.date "year" output.check
    }
    { format.article.crossref output.nonnull
      format.pages output
    }
  if$
  format.issn output
  format.url output
  new.block
  note output
  fin.entry
}

FUNCTION {book}
{ output.bibitem
  author empty$
%<*nat>
    { format.editors "author and editor" output.check
      editor format.key output
    }
    { format.authors output.nonnull
      crossref missing$
        { "author and editor" editor either.or.check }
        'skip$
      if$
    }
%</nat>
%<*plain>
    { format.editors "author and editor" output.check }
    { format.authors output.nonnull
      crossref missing$
        { "author and editor" editor either.or.check }
        'skip$
      if$
    }
%</plain>
  if$
%<*nat>
  new.sentence
  format.year "year" output.check
%</nat>
  new.block
  format.btitle "title" output.check
%<nat>  format.edition output
  crossref missing$
    { format.bvolume output
      new.block
      format.number.series output
      new.sentence
      publisher "publisher" output.check
      address output
    }
    { new.block
      format.book.crossref output.nonnull
    }
  if$
%<plain>  format.edition output
%<plain>  format.date "year" output.check
  format.isbn output
  format.url output
  new.block
  note output
  fin.entry
}

FUNCTION {booklet}
{ output.bibitem
  format.authors output
%<*nat>
  author format.key output
  new.sentence
  format.year "year" output.check
%</nat>
  new.block
  format.title "title" output.check
  howpublished address new.block.checkb
  howpublished output
  address output
%<plain>  format.date output
  format.isbn output
  format.url output
  new.block
  note output
  fin.entry
}

FUNCTION {inbook}
{ output.bibitem
  author empty$
    { format.editors "author and editor" output.check
%<nat>      editor format.key output
    }
    { format.authors output.nonnull
      crossref missing$
        { "author and editor" editor either.or.check }
        'skip$
      if$
    }
  if$
%<*nat>
  new.sentence
  format.year "year" output.check
%</nat>
  new.block
  format.btitle "title" output.check
%<nat>  format.edition output
  crossref missing$
    { format.bvolume output
      format.chapter.pages "chapter and pages" output.check
      new.block
      format.number.series output
      new.sentence
      publisher "publisher" output.check
      address output
    }
    { format.chapter.pages "chapter and pages" output.check
      new.block
      format.book.crossref output.nonnull
    }
  if$
%<plain>  format.edition output
%<plain>  format.date "year" output.check
  format.isbn output
  format.url output
  new.block
  note output
  fin.entry
}

FUNCTION {incollection}
{ output.bibitem
  format.authors "author" output.check
%<*nat>
  author format.key output
  new.sentence
  format.year "year" output.check
%</nat>
  new.block
  format.title "title" output.check
  new.block
  crossref missing$
    { format.in.ed.booktitle "booktitle" output.check
%<nat>      format.edition output
      format.bvolume output
      format.number.series output
      format.chapter.pages output
      new.sentence
      publisher "publisher" output.check
      address output
%<plain>      format.edition output
%<plain>      format.date "year" output.check
    }
    { format.incoll.inproc.crossref output.nonnull
      format.chapter.pages output
    }
  if$
  format.isbn output
  format.url output
  new.block
  note output
  fin.entry
}

FUNCTION {inproceedings}
{ output.bibitem
  format.authors "author" output.check
%<*nat>
  author format.key output
  new.sentence
  format.year "year" output.check
%</nat>
  new.block
  format.title "title" output.check
  new.block
  crossref missing$
    { format.in.ed.booktitle "booktitle" output.check
      format.bvolume output
      format.number.series output
      format.pages output
      address empty$
        { organization publisher new.sentence.checkb
          organization output
          publisher output
%<plain>          format.date "year" output.check
        }
        { address output.nonnull
%<plain>          format.date "year" output.check
%<nat>          format.date output
          new.sentence
          organization output
          publisher output
        }
      if$
    }
    { format.incoll.inproc.crossref output.nonnull
      format.pages output
    }
  if$
  format.isbn output
  format.url output
  new.block
  note output
  fin.entry
}

FUNCTION {conference} { inproceedings }

FUNCTION {manual}
{ output.bibitem
%<*nat>
  format.authors output
  author format.key output
  new.sentence
  format.year "year" output.check
  new.block
  format.btitle "title" output.check
  format.edition output
  organization address new.block.checkb
  organization output
  address output
%</nat>
%<*plain>
  author empty$
    { organization empty$
        'skip$
        { organization output.nonnull
          address output
        }
      if$
    }
    { format.authors output.nonnull }
  if$
  new.block
  format.btitle "title" output.check
  author empty$
    { organization empty$
        { address new.block.checka
          address output
        }
        'skip$
      if$
    }
    { organization address new.block.checkb
      organization output
      address output
    }
  if$
  format.edition output
  format.date output
%</plain>
  format.url output
  new.block
  note output
  fin.entry
}

FUNCTION {mastersthesis}
{ output.bibitem
  format.authors "author" output.check
%<*nat>
  author format.key output
  new.sentence
  format.year "year" output.check
%</nat>
  new.block
  format.title "title" output.check
  new.block
  "Master's thesis" format.thesis.type output.nonnull
  school "school" output.check
  address output
%<plain>  format.date "year" output.check
  format.isbn output
  format.issn output
  format.url output
  new.block
  note output
  fin.entry
}

FUNCTION {misc}
{ output.bibitem
  format.authors output
%<*nat>
  author format.key output
  new.sentence
  format.year "year" output.check
%</nat>
  title howpublished new.block.checkb
  format.title output
  howpublished new.block.checka
  howpublished output
%<plain>  format.date output
  format.isbn output
  format.issn output
  format.url output
  new.block
  note output
  fin.entry
  empty.misc.check
}

FUNCTION {phdthesis}
{ output.bibitem
  format.authors "author" output.check
%<*nat>
  author format.key output
  new.sentence
  format.year "year" output.check
%</nat>
  new.block
  format.btitle "title" output.check
  new.block
  "PhD thesis" format.thesis.type output.nonnull
  school "school" output.check
  address output
%<plain>  format.date "year" output.check
  format.isbn output
  format.issn output
  format.url output
  new.block
  note output
  fin.entry
}

FUNCTION {proceedings}
{ output.bibitem
%<*nat>
  format.editors output
  editor format.key output
  new.sentence
  format.year "year" output.check
  new.block
  format.btitle "title" output.check
  format.bvolume output
  format.number.series output
  address output
  new.sentence
  organization output
  publisher output
%</nat>
%<*plain>
  editor empty$
    { organization output }
    { format.editors output.nonnull }
  if$
  new.block
  format.btitle "title" output.check
  format.bvolume output
  format.number.series output
  address empty$
    { editor empty$
        { publisher new.sentence.checka }
        { organization publisher new.sentence.checkb
          organization output
        }
      if$
      publisher output
      format.date "year" output.check
    }
    { address output.nonnull
      format.date "year" output.check
      new.sentence
      editor empty$
        'skip$
        { organization output }
      if$
      publisher output
    }
  if$
%</plain>
  format.isbn output
  format.url output
  new.block
  note output
  fin.entry
}

FUNCTION {techreport}
{ output.bibitem
  format.authors "author" output.check
%<*nat>
  author format.key output
  new.sentence
  format.year "year" output.check
%</nat>
  new.block
  format.title "title" output.check
  new.block
  format.tr.number output.nonnull
  institution "institution" output.check
  address output
%<plain>  format.date "year" output.check
  format.issn output
  format.url output
  new.block
  note output
  fin.entry
}

FUNCTION {unpublished}
{ output.bibitem
  format.authors "author" output.check
%<*nat>
  author format.key output
  new.sentence
  format.year "year" output.check
%</nat>
  new.block
  format.title "title" output.check
  format.url output
  new.block
  note "note" output.check
%<plain>  format.date output
  fin.entry
}

FUNCTION {default.type} { misc }


MACRO {jan} {"January"}

MACRO {feb} {"February"}

MACRO {mar} {"March"}

MACRO {apr} {"April"}

MACRO {may} {"May"}

MACRO {jun} {"June"}

MACRO {jul} {"July"}

MACRO {aug} {"August"}

MACRO {sep} {"September"}

MACRO {oct} {"October"}

MACRO {nov} {"November"}

MACRO {dec} {"December"}

MACRO {acmcs} {"ACM Computing Surveys"}

MACRO {acta} {"Acta Informatica"}

MACRO {cacm} {"Communications of the ACM"}

MACRO {ibmjrd} {"IBM Journal of Research and Development"}

MACRO {ibmsj} {"IBM Systems Journal"}

MACRO {ieeese} {"IEEE Transactions on Software Engineering"}

MACRO {ieeetc} {"IEEE Transactions on Computers"}

MACRO {ieeetcad}
 {"IEEE Transactions on Computer-Aided Design of Integrated Circuits"}

MACRO {ipl} {"Information Processing Letters"}

MACRO {jacm} {"Journal of the ACM"}

MACRO {jcss} {"Journal of Computer and System Sciences"}

MACRO {scp} {"Science of Computer Programming"}

MACRO {sicomp} {"SIAM Journal on Computing"}

MACRO {tocs} {"ACM Transactions on Computer Systems"}

MACRO {tods} {"ACM Transactions on Database Systems"}

MACRO {tog} {"ACM Transactions on Graphics"}

MACRO {toms} {"ACM Transactions on Mathematical Software"}

MACRO {toois} {"ACM Transactions on Office Information Systems"}

MACRO {toplas} {"ACM Transactions on Programming Languages and Systems"}

MACRO {tcs} {"Theoretical Computer Science"}

READ

FUNCTION {sortify}
{ purify$
  "l" change.case$
}

INTEGERS { len }

FUNCTION {chop.word}
{ 's :=
  'len :=
  s #1 len substring$ =
    { s len #1 + global.max$ substring$ }
    's
  if$
}

%<*nat>
FUNCTION {format.lab.names}
{ 's :=
  s #1 "{vv~}{ll}" format.name$
  s num.names$ duplicate$
  #2 >
    { pop$ etal * }
    { #2 <
        'skip$
        { s #2 "{ff }{vv }{ll}{ jj}" format.name$ "others" =
            { etal * }
            { et * s #2 "{vv~}{ll}" format.name$ * }
          if$
        }
      if$
    }
  if$
}

FUNCTION {author.key.label}
{ author empty$
    { key empty$
        { cite$ #1 #3 substring$ }
        'key
      if$
    }
    { author format.lab.names }
  if$
}

FUNCTION {author.editor.key.label}
{ author empty$
    { editor empty$
        { key empty$
            { cite$ #1 #3 substring$ }
            'key
          if$
        }
        { editor format.lab.names }
      if$
    }
    { author format.lab.names }
  if$
}

FUNCTION {author.key.organization.label}
{ author empty$
    { key empty$
        { organization empty$
            { cite$ #1 #3 substring$ }
            { "The " #4 organization chop.word #3 text.prefix$ }
          if$
        }
        'key
      if$
    }
    { author format.lab.names }
  if$
}

FUNCTION {editor.key.organization.label}
{ editor empty$
    { key empty$
        { organization empty$
            { cite$ #1 #3 substring$ }
            { "The " #4 organization chop.word #3 text.prefix$ }
          if$
        }
        'key
      if$
    }
    { editor format.lab.names }
  if$
}

FUNCTION {calc.short.authors}
{ type$ "book" =
  type$ "inbook" =
  or
    'author.editor.key.label
    { type$ "proceedings" =
        'editor.key.organization.label
        { type$ "manual" =
            'author.key.organization.label
            'author.key.label
          if$
        }
      if$
    }
  if$
  'short.list :=
}

FUNCTION {calc.label}
{ calc.short.authors
  short.list
  "("
  *
  year duplicate$ empty$
  short.list key field.or.null = or
     { pop$ "" }
     'skip$
  if$
  *
  'label :=
}
%</nat>

FUNCTION {sort.format.names}
{ 's :=
  #1 'nameptr :=
  ""
  s num.names$ 'numnames :=
  numnames 'namesleft :=
    { namesleft #0 > }
    { nameptr #1 >
        { "   " * }
        'skip$
      if$
      s nameptr "{vv{ } }{ll{ }}{  ff{ }}{  jj{ }}" format.name$ 't :=
      nameptr numnames = t "others" = and
        { "et al" * }
        { t sortify * }
      if$
      nameptr #1 + 'nameptr :=
      namesleft #1 - 'namesleft :=
    }
  while$
}

FUNCTION {sort.format.title}
{ 't :=
  "A " #2
    "An " #3
      "The " #4 t chop.word
    chop.word
  chop.word
  sortify
  #1 global.max$ substring$
}

FUNCTION {author.sort}
{ author empty$
    { key empty$
        { "to sort, need author or key in " cite$ * warning$
          ""
        }
        { key sortify }
      if$
    }
    { author sort.format.names }
  if$
}

FUNCTION {author.editor.sort}
{ author empty$
    { editor empty$
        { key empty$
            { "to sort, need author, editor, or key in " cite$ * warning$
              ""
            }
            { key sortify }
          if$
        }
        { editor sort.format.names }
      if$
    }
    { author sort.format.names }
  if$
}

FUNCTION {author.organization.sort}
{ author empty$
    { organization empty$
        { key empty$
            { "to sort, need author, organization, or key in " cite$ * warning$
              ""
            }
            { key sortify }
          if$
        }
        { "The " #4 organization chop.word sortify }
      if$
    }
    { author sort.format.names }
  if$
}

FUNCTION {editor.organization.sort}
{ editor empty$
    { organization empty$
        { key empty$
            { "to sort, need editor, organization, or key in " cite$ * warning$
              ""
            }
            { key sortify }
          if$
        }
        { "The " #4 organization chop.word sortify }
      if$
    }
    { editor sort.format.names }
  if$
}

FUNCTION {presort}
{
%<*nat>
  calc.label
  label sortify
  "    "
  *
%</nat>
  type$ "book" =
  type$ "inbook" =
  or
    'author.editor.sort
    { type$ "proceedings" =
        'editor.organization.sort
        { type$ "manual" =
            'author.organization.sort
            'author.sort
          if$
        }
      if$
    }
  if$
  "    "
  *
  year field.or.null sortify
  *
  "    "
  *
  title field.or.null
  sort.format.title
  *
  #1 entry.max$ substring$
%<*nat>
  'sort.label :=
  sort.label *
  #1 entry.max$ substring$
%</nat>
  'sort.key$ :=
}

ITERATE {presort}

SORT

%<*nat>
STRINGS { longest.label last.label next.extra }

INTEGERS { longest.label.width last.extra.num number.label }

FUNCTION {initialize.longest.label}
{ "" 'longest.label :=
  #0 int.to.chr$ 'last.label :=
  "" 'next.extra :=
  #0 'longest.label.width :=
  #0 'last.extra.num :=
  #0 'number.label :=
}

FUNCTION {forward.pass}
{ last.label label =
    { last.extra.num #1 + 'last.extra.num :=
      last.extra.num int.to.chr$ 'extra.label :=
    }
    { "a" chr.to.int$ 'last.extra.num :=
      "" 'extra.label :=
      label 'last.label :=
    }
  if$
  number.label #1 + 'number.label :=
}

FUNCTION {reverse.pass}
{ next.extra "b" =
    { "a" 'extra.label := }
    'skip$
  if$
  extra.label 'next.extra :=
  extra.label
  duplicate$ empty$
    'skip$
    { "{\natexlab{" swap$ * "}}" * }
  if$
  'extra.label :=
  label extra.label * 'label :=
}

EXECUTE {initialize.longest.label}

ITERATE {forward.pass}

REVERSE {reverse.pass}

FUNCTION {bib.sort.order}
{ sort.label  'sort.key$ :=
}

ITERATE {bib.sort.order}

SORT
%</nat>
%<*plain>
STRINGS { longest.label }

INTEGERS { number.label longest.label.width }

FUNCTION {initialize.longest.label}
{ "" 'longest.label :=
  #1 'number.label :=
  #0 'longest.label.width :=
}

FUNCTION {longest.label.pass}
{ number.label int.to.str$ 'label :=
  number.label #1 + 'number.label :=
  label width$ longest.label.width >
    { label 'longest.label :=
      label width$ 'longest.label.width :=
    }
    'skip$
  if$
}

EXECUTE {initialize.longest.label}

ITERATE {longest.label.pass}
%</plain>

FUNCTION {begin.bib}
{ preamble$ empty$
    'skip$
    { preamble$ write$ newline$ }
  if$
%<*nat>
  "\begin{thebibliography}{" number.label int.to.str$ * "}" *
  write$ newline$
  "\expandafter\ifx\csname natexlab\endcsname\relax"
  write$ newline$
  "  \def\natexlab#1{#1}\fi" write$ newline$
%</nat>
%<*plain>
  "\begin{thebibliography}{"  longest.label  * "}" *
  write$ newline$
%</plain>
  "\expandafter\ifx\csname url\endcsname\relax"
  write$ newline$
  "  \def\url#1{\texttt{#1}}\fi" write$ newline$
}

EXECUTE {begin.bib}

EXECUTE {init.state.consts}

ITERATE {call.type$}

FUNCTION {end.bib}
{ newline$
  "\end{thebibliography}" write$ newline$
}

EXECUTE {end.bib}
%</bstfile>
%    \end{macrocode}
%
%
% \section{Documentation Driver}
%
%    \begin{macrocode}
%<*driver>
\documentclass[a4paper]{article}
\usepackage[T1]{fontenc}
\usepackage{textcomp}
\usepackage{lmodern}
\usepackage[latin1]{inputenc}
\usepackage{doc}
\usepackage{ifpdf}
\ifpdf
  \RequirePackage[a4paper,bookmarksopen=true,%
                  pdfauthor={Lars Engebretsen},%
                  pdftitle={LaTeX Classes for KTH Theses},%
                  pdfpagemode=None,pdfstartview=FitH]{hyperref}%
\else
  \RequirePackage[T1]{url}
\fi
\def\UrlLeft{\mbox{«}}
\def\UrlRight{\mbox{»}}
\urlstyle{rm}
\def\path{\begingroup\Url}
\newcommand{\bst}[1]{\mbox{«#1»}}
\newcommand{\class}[1]{\mbox{«#1»}}
\newcommand{\command}[1]{\mbox{«#1»}}
\newcommand{\file}[1]{\mbox{«#1»}}
\newcommand{\option}[1]{\mbox{«#1»}}
\newcommand{\style}[1]{\mbox{«#1»}}
\AtBeginDocument{\MakeShortVerb{\|}}
\setcounter{StandardModuleDepth}{1}
\OnlyDescription
\title{LaTeX Classes for KTH Theses}
\date{2005-04-14}
\author{Lars Engebretsen}
\begin{document}
\maketitle
\DocInput{kthesis.dtx}
\end{document}
%</driver>
%    \end{macrocode}
%
