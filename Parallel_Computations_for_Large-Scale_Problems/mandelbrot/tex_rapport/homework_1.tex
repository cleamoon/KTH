\documentclass[a4paper]{article}
\usepackage{titlesec}
\usepackage{fullpage} % Package to use full page
\usepackage{parskip} % Package to tweak paragraph skipping
\usepackage{tikz} % Package for drawing
\usepackage{amsmath}
\usepackage{hyperref}


\titleformat{\section}
  {\normalfont\Large\bfseries}   % The style of the section title
  {}                             % a prefix
  {0pt}                          % How much space exists between the prefix and the title
  {Problem \thesection:\quad}    % How the section is represented

 
\title{SF2568 Homework 1}
\author{Yue Jiao}
\date{}

\begin{document}

\maketitle

\section{} % 1
A processor is a physical hardware device that has the capability of accessing the memory and computing values.
A process is a computational activity assigned to a processor that do the computation.

\section{} % 2
We assume that there is totally P floating operations to compute.
According to the problem, $10\%P$ of operations shall be computed sequentially and $90\%P$ of operations shall be computed parallelly on 100 pressors.
Then we assume that it will take a time $t$ to compute this.
So, we will have the following relation:
\[
\frac{10\%P}{2\text{G flops}} + \frac{90\%P}{100\times2\text{G flops}} = t
\]
The average performance can be calculated by $\frac{P}{t}$. So:
\[
\frac{P}{t}=\frac{P}{\frac{10\%P}{2\text{G flops}} + \frac{90\%P}{100\times2\text{G flops}}} \approx 18.3\text{G flops}
\]


\section{} % 3
The system efficiency $\eta_P$ is defined as the kvot of the systems parallel speedup $S_p$ and the amount of processors $P$, $\frac{S_p}{P}$.
The theoretical maximum of parallel speedup $S_p$ is $P$ so the kvot has the theoretical maximum of $1$.
So there is no way the system efficiency to be greater than $100\%$.

\section{} % 4
Since the computation part of the rank sort algorithm is completely independent from each number, we can let each processor compute the rank of $10$ numbers one after one. 
So we first divide the set of number that need to be sorted in $P$ subset where all set contains 10 numbers and $P$ is the number of processors.
Then we send the mark of each set of numbers to different processors so they know which numbers they need to compute on.
Then each processor shall calculate ranks of their own 10 numbers one after one and send the results back.
The complete result shall contain the ranks of all numbers.

\section{} % 5
\textbf{a)}
Let's assume that the fraction of serial running time is $f$.
So, after the parallelization, the time spended in the serial section shall be $64\text{s} \times f$.
The time spended on the parallel section shall be $\frac{64\text{s} \times \left(1-f\right)}{8}$.
Totally the parallel running time will be:
\[
64\text{s} \times f + \frac{64\text{s} \times \left(1-f\right)}{8} = 22\text{s} \Rightarrow f=25\%
\]
So $25\%$ of serial running time is spended in the serial section.
\\

\textbf{b)}
The parallel speedup is simply
\[
S_p = \frac{64s}{22s} = 2.91
\]

\textbf{c)}
Let's assume that the serial runnning time is $t$.
So, the phases consume $20\%t$, $20\%t$ and $60\%t$ of time.
Since we have $15$ processors, we can assume that each phase can run optimally.
The time spended in the first phase is then $\frac{20\%t}{5}$. 
The time spended in the second phase is then $\frac{20\%t}{10}$. 
The time spended in the third phase is then $\frac{60\%t}{15}$.
So, the total running parallel runnning time shall be:
\[
\frac{20\%t}{5} + \frac{20\%t}{10} + \frac{60\%t}{15} = \frac{t}{10}
\]
Thus, the system parallel speedup is then:
\[
S_P=\frac{t}{\frac{t}{10}} = 10
\]

\end{document}
              
