\subsection{Original problem}
\subsubsection*{Definitions}


As the problem description indicates above, the entire power system contains 6 generators and 4 nodes. The nodes are connected by power lines between them. These objects are described as set in the following way:
\begin{itemize}
    \item $\mathcal{I} = \{1,2,3,4,5,6\}$ where $i$ corresponds to the $i$-th generator,
    \item $\mathcal{K}= \{1,2,3,4\}$ where $k$ corresponds to the $k$-th node,
    \item $\mathcal{M}= \{1,2,3,4\}$ where $m$ corresponds to the $m$-th node.
\end{itemize}
Two node sets are needed to symbolise the interconnection.

\vspace{3mm}
Several more parameters are needed to describe the problem. 
The parameters, for which the values are fixed and given in the Appendix, are the following:
\begin{align*}
\textbf{Symbol} \enspace &: \textbf{Definition}\\
Cap_i &: \ \text{the maximum capacity of generator } i, \\
Cost_i &: \ \text{the cost of producing } 1 pu \text{ of power during } 1 h \text{ by generator } i, \\
D_k &: \ \text{the demand of active power on the node } k, \\
g_{k,m} &: \ \text{parameter between the link } (k,m), \\
b_{k,m} &: \ \text{parameter between the link } (k,m).
\end{align*}
The two last parameters are used by the equations which give the active power and reactive power transmission between the nodes. 

\vspace{3mm}
To solve the problem we have to find the optimal values of several variables. Each generator can generate a certain amount of active power and a certain amount of reactive power. The voltage at each node is described by two variables, the voltage amplitude and the voltage phase angle. The variables are the following:
\begin{align*}
\textbf{Symbol} \enspace &: \textbf{Definition}\\
P_{G_i} &: \ \text{the active power generated by generator } i, \\
Q_{G_i} &: \ \text{the reactive power generated/absorbed by generator } i, \\
U_k &: \ \text{the voltage amplitude of the node } k, \\
\theta_k &: \ \text{the voltage phase angle of the node } k. 
\end{align*}

\vspace{3mm}
Two functions, derived from the electromagnetic theory, are introduced. These functions provide the mathematical connection between the power flows and the voltages between nodes: 
\begin{align*}
\textbf{Symbol} \enspace &: \textbf{Definition} \enspace = \textbf{Value}\\
P_{k,m} &: \ \text{the active power transmission between the link } (k,m), \\
        &= U^2_{k} g_{km} − U_{k} U_{m} g_{km} \cos(\theta_k − \theta_m) − U_k U_m b_{km} \sin(\theta_k − \theta_m);\\
Q_{k,m} &: \ \text{the reactive power transmission between the link } (k,m), \\
        &= −U^2_k b_{km} + U_k U_m b_{km} \cos(\theta_k − \theta_m) − U_k U_m g_{km} \sin(\theta_k − \theta_m).
\end{align*}

\vspace{3mm}
The variables defined above are bounded. The definition region of the variables are modelled by the following constraints: 
\begin{itemize}
    \item The active power generated by each generator is limited:
    \begin{equation*}
        0 \leq P_{G_i} \leq Cap_i  \qquad \forall i \in \mathcal{I}.
    \end{equation*}
    
    \item The reactive power generated/absorbed by each generator is limited:
    \begin{equation*}
        -Cap_i \leq Q_{G_i} \leq Cap_i  \qquad \forall i \in \mathcal{I}.
    \end{equation*}
    
    \item The voltage amplitude for each node is limited:
    \begin{equation*}
        0.9 \leq U_k \leq 1.1  \qquad \forall k \in \mathcal{K}.
    \end{equation*}
    
    \item The voltage phase angle for each node is limited:
    \begin{equation*}
        -\pi \leq \theta_k \leq \pi  \qquad \forall k \in \mathcal{K}.
    \end{equation*}
\end{itemize}

\vspace{3mm}
On each node, there are requirements of flow-balance for the active and the reactive power. \\
To find the constraint, we observe that: 1) It is assumed that the voltage amplitude $U_k$ on each node is a constant value, 2) It is known from the electromagnetic theory that the electric power is given by $P_k = U_k\times I_k$ and 3) According to the Kirchhoff's current law, \textit{the sum of the current flowing into a node is equal to the sum of current flowing out of that node}. \\
From those three observations, we deduce that \textit{the sum of the power flowing into a node has to be equal to the sum of the power flowing out of that node}. Those requirements gives the following two flow-balance constraints: 
\begin{itemize}
    \item The flow-balance for the active power at each node:
    \begin{align*}
        \overbrace{I[k = 1] \cdot \sum_{i = 1}^{2} P_{G_i} + I[k = 2] \cdot \sum_{i = 3}^{6} P_{G_i} + \sum_{m \in \mathcal{M}} P_{m,k}}^{\text{Power flowing into the node }k} = \overbrace{ D_k + \sum_{m \in \mathcal{M}} P_{k,m} }^{\text{Power flowing out of the node }k} &\qquad \forall k \in \mathcal{K},\\
        I[k = 1] \cdot \sum_{i = 1}^{2} P_{G_i} + I[k = 2] \cdot \sum_{i = 3}^{6} P_{G_i} + \sum_{m \in \mathcal{M}} \big( P_{m,k} - P_{k,m} \big) = D_k &\qquad \forall k \in \mathcal{K},
    \end{align*}
    where $I[k=j]$ is equal to $1$ if $k$ is equal to $j$ and $0$ otherwise.
    
    \item The flow-balance for the reactive power at each node:
    \begin{equation*}
        I[k = 1] \cdot \sum_{i = 1}^{2} Q_{G_i} + I[k = 2] \cdot \sum_{i = 3}^{6} Q_{G_i} + \sum_{m \in \mathcal{M}} \big( Q_{m,k} - Q_{k,m} \big) = 0 \qquad \forall k \in \mathcal{K}.
    \end{equation*}
\end{itemize}

\vspace{3mm}
The goal is to minimise the power production cost per hour of the system. The cost of the system is given by the active power generation of each generator multiplied with the cost of running the corresponding generator. Thus, the objective function is the following:
\begin{equation*}
    z = \min_{P_{G_i}, Q_{G_i}, U_k, \theta_k} \sum_{i \in \mathcal{I}} P_{G_i} \cdot Cost_i.
\end{equation*}

\subsubsection*{To summarize}
Here below is the summary of the formulation of the nonlinear model: 
\begin{alignat*}{2}
&\min_{P_{G_i}, Q_{G_i}, U_k, \theta_k} \sum_{i \in \mathcal{I}} P_{G_i} \cdot Cost_i\\
&\\
&\text{subject to:}\nonumber \\
&0 \leq P_{G_i} \leq Cap_i  &\qquad \forall i \in \mathcal{I}\\
&-Cap_i \leq Q_{G_i} \leq Cap_i  &\qquad \forall i \in \mathcal{I}\\
&  0.9 \leq U_k \leq 1.1  &\qquad \forall k \in \mathcal{K}\\
&  -\pi \leq \theta_k \leq \pi  &\qquad \forall k \in \mathcal{K}\\
&  I[k = 1] \cdot \sum_{i = 1}^{2} P_{G_i} + I[k = 2] \cdot \sum_{i = 3}^{6} P_{G_i} + \sum_{m \in \mathcal{M}} \big( P_{m,k} - P_{k,m} \big) = D_k &\qquad \forall k \in \mathcal{K} \\
& I[k = 1] \cdot \sum_{i = 1}^{2} Q_{G_i} + I[k = 2] \cdot \sum_{i = 3}^{6} Q_{G_i} + \sum_{m \in \mathcal{M}} \big( Q_{m,k} - Q_{k,m} \big) = 0 &\qquad \forall k \in \mathcal{K},\\
&\\
&\text{where:}\nonumber \\
& P_{k,m} = U^2_{k} g_{km} − U_{k} U_{m} g_{km} \cos(\theta_k − \theta_m) − U_k U_m b_{km} \sin(\theta_k − \theta_m) \\
& Q_{k,m} = −U^2_k b_{km} + U_k U_m b_{km} \cos(\theta_k − \theta_m) − U_k U_m g_{km} \sin(\theta_k − \theta_m).
\end{alignat*} 

\subsubsection*{Non-convexity}
The problem is not convex. Indeed, the variables $P_{G_i}$ in the objective have non-linear equality constraints involving square, sinus and cosinus functions. These kind of constraints often leads to non-convex problems since the resulting feasible region is usually not convex. %has a form of curved line. <-- special case, no ?

\newpage

\subsection{Approximation of the OPF}
\subsubsection*{Definitions}
%To simplify the calculation, approximations are often needed to transform the nonlinear programming problem into a linear programming problem. This is done in this part of rapport.
To simplify the calculations, three approximations will be used to transform our nonlinear problem into a linear problem. The approximations are the following:
\begin{enumerate}
    \item \textit{$U_k = 1$ for all nodes.} We noticed that the definition domain of the voltage amplitude is quite narrow, only from $0.9 pu$ to $1.1 pu$. So it is appropriate to assume that $U_k = 1 pu$ for all node $k$.
    
    \item \textit{$\sin{(\theta_k - \theta_m)} \approx \theta_k - \theta_m$ and $\cos{(\theta_k - \theta_m)} \approx 1$ for all links.} Observing the results of the nonlinear problem, we can notice that the phase angle difference, $\theta_k - \theta_m$, differ less than $0.05 \ rad$. So the trigonometrical functions can be approximated.
    \item \textit{Disregards the reactive power.} Still observing the results of the nonlinear programming question, we can notice a second thing. The reactive power flows are relatively small compared with the active power flows on each link. Thus we could simply ignore them.
\end{enumerate}

\vspace{3mm}
\noindent 
The sets are the same:
\begin{itemize}
    \item $\mathcal{I} = \{1,2,3,4,5,6\}$ where $i$ corresponds to the i-th generator,
    \item $\mathcal{K}= \{1,2,3,4\}$ where $k$ corresponds to the k-th node,
    \item $\mathcal{M}= \{1,2,3,4\}$ where $m$ corresponds to the m-th node.
\end{itemize}

\vspace{3mm}
\noindent
The parameters $g_{k,m}$ are not used anymore, the other parameters are the same:
\begin{align*}
\textbf{Symbol} \enspace &: \textbf{Definition}\\
Cap_i &: \ \text{the maximum capacity of generator } i, \\
Cost_i &: \ \text{the cost of producing } 1 pu \text{ of power during } 1 h \text{ by generator } i, \\
D_k &: \ \text{the demand of active power on the node } k, \\
b_{k,m} &: \ \text{parameter between the link } (k,m).
\end{align*}

\vspace{3mm}
\noindent
Since we set the values of $U_k$ to $1$ and we disregard the reactive power $Q_{G_i}$ the only variables left are:
\begin{align*}
\textbf{Symbol} \enspace &: \textbf{Definition}\\
P_{G_i} &: \ \text{the active power generated by generator } i, \\
\theta_k &: \ \text{the voltage phase angle of the node } k. 
\end{align*}

\vspace{3mm}
\noindent
The function for the active power is still used but greatly simplified, while the function for the reactive power is simply dismissed:
\begin{align*}
\textbf{Symbol} \enspace &: \textbf{Definition} \enspace = \textbf{Value}\\
P_{k,m} &: \ \text{the active power transmission between the link } (k,m), \\
        &= − b_{km} (\theta_k − \theta_m);\\
\end{align*}

\newpage
\noindent
The remaining constraints are: 
\begin{itemize}
    \item The active power generated by each generator is limited:
    \begin{equation*}
        0 \leq P_{G_i} \leq Cap_i  \qquad \forall i \in \mathcal{I}.
    \end{equation*}
    
    \item The voltage phase angle for each node is limited:
    \begin{equation*}
        -\pi \leq \theta_k \leq \pi  \qquad \forall k \in \mathcal{K}.
    \end{equation*}
    
    \item The flow-balance for the active power at each node:
    \begin{equation*}
        I(k = 1) \cdot \sum_{i = 1}^{2} P_{G_i} + I(k = 2) \cdot \sum_{i = 3}^{6} P_{G_i} + \sum_{m \in \mathcal{I}} \big( P_{m,k} - P_{k,m} \big) = D_k \qquad \forall k \in \mathcal{K}.
    \end{equation*}
    
\end{itemize}

\vspace{3mm}
\noindent
The objective function is the same but the minimization only depends on the remaining variables:
\begin{equation*}
    z = \min_{P_{G_i}, \theta_k} \sum_{i \in \mathcal{I}} P_{G_i} \cdot Cost_i.
\end{equation*}


\subsubsection*{To summarise}
\begin{alignat*}{2}
&\min_{P_{G_i}, \theta_k} \sum_{i \in \mathcal{I}} P_{G_i} \cdot Cost_i\\
&\\
&\text{subject to:}\nonumber \\
&0 \leq P_{G_i} \leq Cap_i  &\qquad \forall i \in \mathcal{I}\\
&  -\pi \leq \theta_k \leq \pi  &\qquad \forall k \in \mathcal{K}\\
&  I[k = 1] \cdot \sum_{i = 1}^{2} P_{G_i} + I[k = 2] \cdot \sum_{i = 3}^{6} P_{G_i} + 2 \sum_{m \in \mathcal{M}} b_{km} (\theta_k − \theta_m)  = D_k &\qquad \forall k \in \mathcal{K} 
\end{alignat*} 

\subsubsection*{Convexity}
The problem is convex. Indeed, the original non-linear problem is approximated by a linear problem which is always convex. 
